% !TEX encoding = IsoLatin2
\documentclass[Bachelor, BIC, german]{twbook}

\usepackage[T1]{fontenc}
% Hier kann je nach Betriebssystem eine der folgenden Optionen notwendig sein, 
% um die Umlaute korrekt wiederzugeben: utf8, latin, applemac
\usepackage[utf8]{inputenc}

\usepackage{lipsum}
\setlipsumdefault{1-4}

\usepackage{comment}
\usepackage{hyperref}

% Einstellungen für Code-Listings
\usepackage{listings}
\usepackage{color}

\definecolor{mygreen}{rgb}{0,0.6,0}
\definecolor{mygray}{rgb}{0.5,0.5,0.5}
\definecolor{mymauve}{rgb}{0.58,0,0.82}

\lstset{ 
  backgroundcolor=\color{white},   % choose the background color; you must add \usepackage{color} or \usepackage{xcolor}; should come as last argument
  basicstyle=\small\ttfamily,        % the size of the fonts that are used for the code
  breakatwhitespace=false,         % sets if automatic breaks should only happen at whitespace
  breaklines=true,                 % sets automatic line breaking
  captionpos=b,                    % sets the caption-position to bottom
  commentstyle=\color{mygreen},    % comment style
  frame=single,	                   % adds a frame around the code
  keepspaces=true,                 % keeps spaces in text, useful for keeping indentation of code (possibly needs columns=flexible)
  keywordstyle=\color{blue},       % keyword style
  numbers=left,                    % where to put the line-numbers; possible values are (none, left, right)
  numbersep=5pt,                   % how far the line-numbers are from the code
  numberstyle=\tiny\color{mygray}, % the style that is used for the line-numbers
  rulecolor=\color{black},         % if not set, the frame-color may be changed on line-breaks within not-black text (e.g. comments (green here))
  showspaces=false,                % show spaces everywhere adding particular underscores; it overrides 'showstringspaces'
  showstringspaces=false,          % underline spaces within strings only
  showtabs=false,                  % show tabs within strings adding particular underscores
  stepnumber=1,                    % the step between two line-numbers. If it's 1, each line will be numbered
  tabsize=2,	                   % sets default tabsize to 2 spaces
  title=\lstname                   % show the filename of files included with \lstinputlisting; also try caption instead of title
}

\graphicspath{ {./PICs/} }
\DeclareGraphicsExtensions{.pdf,.png}

% Einstellungen der Class TWBOOK
\title{DNS Client Security\\Bedrohungsszenarien und Lösungswege}
\author{Michael Riedmann}
\studentnumber{1510258054}
\supervisor{FH-Prof. Dipl.-Ing. Alexander Mense}
\place{Wien}

\kurzfassung{% Wird vom Template eingefügt, kein Chapter oder so einfügen!

Zusammenfassung

\todo[inline]{Alles andere schon fertig? Wenn nein, geh weg!}}
% TODO: Schlagwörter der Zusammenfassung ausfüllen
\schlagworte{Schlagwort1, Schlagwort2, Schlagwort3, Schlagwort4}

\outline{%% spellcheck-language "en"

% Wird vom Template eingefügt, kein Chapter oder so einfügen!

Abstract

\todo[inline]{Alles andere schon fertig? Wenn nein, geh weg!}}
% TODO: Schlagwörter des Abstracts ausfüllen
\keywords{Keyword1, Keyword2, Keyword3, Keyword4}

% --------------------------------------------------------------------------- %
\begin{document}
\maketitle

\chapter{Einleitung}
Das Domain-Name-System (DNS) bietet eine einfache Möglichkeit zur Namesauflösung und damit die Grundlage für menschenles- und merkbare Namen in modernen Computernetzwerken. Außerdem dient es als verteile Datenbank für simple Informationen über Netzwerke und Hosts. Speziell im Internet nimmt dieses Service damit eine zentrale Rolle im Verbindungsaufbau zwischen den vernetzen Systemen ein. Betrachtet man nun das Netzwerkprotokoll des DNS genauer, stellt man fest, dass es ohne jegliche Ansprüche an Informationssicherheit konzipiert wurde. Dieser Umstand ist dem Alter, beziehungsweise der Historie des Systems geschuldet, entspricht den heutigen IT-Sicherheitsstandards jedoch in keiner Weise. Aufgrund dieser Tatsache, wurden in den letzten Jahrzehnten verschiedenste Ansätze zur Lösung dieses Problems entwickelt. Trotz dieses Umstands hat es bis zum heutigen Tag keiner der entwickelten Standards geschafft eine weitreichende Durchdringung zu erlangen. Obwohl das Bedürfnis nach Sicherheit über die Zeit stark gestiegen ist, trägt DNS aktuell eher zur Verschärfung der Lage als zu dessen Befriedigung bei.

Diese Arbeit bietet in Abschnitt \ref{chap:dns} eine kurze Einführung in die Funktionsweise von DNS und gibt eine Einführung in sicherheitsrelevanten Aspekten des Systems (\ref{sec:dnssecurity}). Kern dieser Arbeit stellt eine detaillierte Darstellung der aktuellen Sicherheitsprobleme und mögliche Lösungen, mit Fokus auf Endgeräte und -nutzer, dar. Dazu werden, in Kapitel \ref{chap:threads}, die wichtigsten Bedrohungen dargelegt. Um diese besser zu verstehen, werden die häufigsten Angriffsmethoden (in Kapitel \ref{chap:attacks}) beschrieben. Anhand dieser konnten Ziele für Mitigationsmöglichkeiten (in Kapitel \ref{chap:goals}) vorgeschlagen werden, welche in den in Kapitel \ref{chap:solutions} festgelegten Lösungskonzepten resultieren. Abschließend werden diese Konzepte mit aktuelle verfügbaren Technologien (\ref{chap:technologies}) verglichen. Die Gesamtheit der erarbeiteten Informationen fließt in einem, unter \ref{chap:implementation} Beschriebenen, Testaufbau zusammen. Dieser soll einen praktischen Ansatz zur bestmöglichen Erfüllung der Formulierten Ziele unter Zuhilfenahme bestehenden Technologien darstellen.


\chapter{Verwandte Arbeiten}

\lipsum

% DONT! JUST DONT! Wenn was streichen das das!
% Is aber wichtig ... oder?

\chapter{Schwachstellen}
\label{cap:weaknesses}

\chapter{Bekannte Probleme}
\label{chap:threads}

\section{Fehlende Integritäts- und Authentizitätsprüfung}
\label{sec:Thread-Auth}

Wie in den Anfängen des World-Wide-Webs und dessen Transportprotokoll HTTP ist auch bei der Konzeptionierung der DNS Protokolls keine Rücksicht auf Sicherheitsaspekte wie Integrität, Authentizität und Vertraulichkeit genommen worden. Da DNS nach RFC1035\cite{rfc1035} auf jede Form von Authentifizierung verzichtet, ist auch ein Prüfen der Identität der jeweiligen Gegenstelle, sowie der eigentlichen Nutzdaten, nicht vorgesehen. Dies birgt eine hohe Anfälligkeit auf Angriffe die Anfragen und Antworten gezielt verändern. Des weiterem besteht keine Möglichkeit die Quellenauthentizität der RR zu prüfen. Damit können Angreifer gefälschte Einträge in die DNS Datenbank einbringen und so die Adresse bestimmter Servicenamen gezielt verändern. Diese Umleitung kann dann als Basis für umfangreiche Angriffe genutzt werden.

\section{Fehlender Schutz der Vertraulichkeit}
\label{sec:Thread-Priv}

Das Internet als globale, vernetzende Technologie, rückt immer weiter in den Mittelpunkt der alltäglichen Kommunikation. Damit verbunden ist der steigende Bedarf nach Schutz der übertragenen Daten. Die IETF behandelt dieses Thema in verschiedensten RFCs, wobei Privacy einen der Schwerpunkte darstellt. Mit der 2013 veröffentlichten RFC6973 \textit{Privacy Considerations for Internet Protocols}\cite{rfc6973} wurde der Grundstein für die aktive Förderung von vertraulicher Kommunikation im Internet gelegt . 
Für DNS wurde die darauf aufbauenden RFC7626 \textit{DNS Privacy Considerations}\cite{rfc7626} erstellt, welche die Relevanz von DNS Privacy klar darlegt. Im speziellen wird auf die notwendige Unterscheidung zwischen der freien Zugänglichkeit der DNS Daten an sich und der Öffentlichkeit der Anfragen aufmerksam gemacht. Da DNS in im ursprünglichen Konzept nicht unterscheidet, sind die Anfragedaten für alle Zwischenstellen lesbar. Darüber hinaus existiert kein Schutz der Vertraulichkeit der Übertragung, dies macht auch Unberechtigten das mitlesen der Kommunikation einfach möglich. 
Diese Umstände werden durch die in RFC7871\cite{RFC7871} spezifizierte Technik \textit{EDNS0 Client Subnet} zusätzlich verschärft. Mit dieser Technologie können sensibler Informationen, wie das Client Subnetz, auch durch Resolver hindurch weitergereicht werden. Wie in der \textit{Privacy Note} der RFC beschrieben sind solche Vorhaben jedoch mit größter Vorsicht und auf keinen Fall gegen den Willen der Nutzenden einzusetzen. Der 2017 eingebrachte Entwurf zum einführen einer ClientID\cite{Licht2017} wurde zur Unterstützung des Privacy-Gedanken nicht weiter verfolgt.

\section{Angreifbarkeit der Verfügbarkeit}
\label{sec:Thread-DosAmp}

DNS ist seit Jahren das meist angegriffene Internet-Service weltweit und befindet sich darüber hinaus auch als Trägertechnologie von DoS-Attacken auf Platz 1 \cite{Alcoy2017}. Das fehlen eines Handshakes beim Verbindungsaufbau und die schlechte Zurückverfolgbarkeit machen DNS, neben dem \textit{Network Time Protocol} (NTP), zur beliebtesten Verstärkungsmethoden für DoS-Angriffe. Da DNS-Server in den meisten Fällen von anderen DNS-Servern angegriffen werden, bedroht diese Anfälligkeit die Verfügbarkeit des Systems selbst, sowie anderer Systeme im Internet \cite{Kambourakis2008}. Es besteht darüber hinaus das Problem, dass die mit DNSSEC eingeführte Erweiterung EDNS0, sowie DNSSEC selbst, das Verstärkungspotenzial weiter erhöht \cite{Anagnostopoulos2013}\cite{VanRijswijk-Deij2014}.


\chapter{Angriffe}
\label{chap:attacks}

Die meisten DNS-Sicherheitsrichtlinien richten sind an Betreiber von extern gerichteten Installationen. Das Thema \textit{DNS Client Security} wird wenn nur am Rande behandelt. Der nachfolgende Abschnitt befasst sich daher von allem mit Bedrohungen der Interaktion zwischen DNS-Resolver und rekursivem DNS-Server. 
Es gibt dabei zwei grundlegende Möglichkeiten einen DNS-Client anzugreifen: Über direktes kommunizieren mit der Netzwerkschnittstelle des Clients oder mittels Angriff auf einen Teil der Client-nahen Infrastruktur. Um die Auswirkung klar ersichtlich zu machen wird in Abschnitt \ref{sec:Attacks-Summary} eine tabellarische Zusammenfassung mit den verletzen Sicherheitskriterien (nach CIA-Triade) gegeben. 

% Trennung wirklich sinnvoll?

\section{Direkte Angriffe}

Angriffe die das Verhalten einzelner DNS-Clients über eingreifen in deren Kommunikationsfluss beeinflussen, können als \textit{direkte Angriffe} zusammenfasst werden. Diese Verfahren sind nur effizient wenn der Angreifer entweder passiv (Sniffing) oder aktiv (Man-in-the-middle) an der Netzwerkverbindung des Clients beteiligt ist. 

\subsection{DNS Sniffing und Spoofing}
% Sniffing / Spoofing zusammenlegen?

Als Sniffing Angriffe werden, nach CAPEC, alle Arten von Angriffen bezeichnet die es ermöglichen Nachrichten zwischen mindestens 2 Parteien zu beobachten, mitlesen und/oder mithören. (https://capec.mitre.org/data/definitions/157.html). Im Kontext DNS hat dies eine spezielle Bedeutung da es, aufgrund des fehlenden Schutz der Vertraulichkeit, ermöglicht, alle Informationen der Anfragen und Antworten einzusehen. Somit ist das Kommunikationsverhalten der Opfergeräte leicht nachzuvollziehen, was zum Beispiel den Recon-Schritt der Cyber-Kill-Chain (siehe \ref{sec:dnssecurity}) erheblich erleichtert. 

In IP-Netzen gibt es verschieden Möglichkeiten eine Sniffing Attacke durchzuführen: 

Eine Möglichkeit ist der Angriff der Netzwerkhardware. Bei einem Hub als Netzwerkverteiler erhalten immer alle angeschlossenen Teilnehmer alle Pakete. Es können daher die Pakete aller anderen an diesem Hub angeschlossenen Geräte einfach mitgelesen werden. Wird ein Switch einsetzt wird für jeden Anschluss (Port) die MAC-Adresse des angeschlossenen Geräts in eine Tabelle eingetragen. Es werden somit nur noch Pakete mit passender MAC-Adresse an die entsprechenden Ports weitergeleitet. Ein Angreifer kann bestimme Switches jedoch durch künstliche Füllen der MAC-Switching-Tabelle (MAC-Flooding Angriff) in ein, einem Hub ähnliches, Verhalten zwingen. Soll ein spezielles Endgerät mit bekannter MAC-Adresse angegriffen werden, kann bei anfälligen Switches auch ein MAC-Duplication Angriff durchgeführt werden. Dabei sendet der Angreifer Pakete mit der gefälschter MAC-Adresse des Opfers was den Switch dazu verleitet die Adresse auf zwei Ports einzutragen. Anfällige Switches senden daraufhin die Pakete des Opfers auch auf den Port des Angreifers.

Ist die Netzwerkhardware nicht angreifbar bleibt noch die Möglichkeit den Netzwerkstack der Client-Geräte zu attackieren. Hier kann eine Schwäche im \textit{Address Resolution Protocol} (ARP) ausgenützt werden. Dieses Protokoll ist für die Auflösung von logischen Adressen (z.B. IP-Adressen) zu Adressen der Hardware (MAC-Adresse) verantwortlich (https://tools.ietf.org/html/rfc826). Da ARP stateless konzeptioniert wurde und auch keine Art von Authentifizierung verlangt kann eine als \textit{ARP Spoofing} bekannte Attacke durchgeführt werden. Mit dieser kann die Zuordnung zwischen MAC-Adresse und logischer Adresse (z.B. IP-Adresse) im lokalen Netzwerk bewusst manipuliert werden. Damit ist es einem Angreifer leicht möglich sich als vertrauenswürdiger Host des Netzwerks auszugeben. Wird die MAC-Adresse des Netzwerk-Gateways mit der eines abhörenden Rechners getauscht ist auch ein umfangreiches Abhören des Netzwerkverkehrs möglich.

% Da es Ethernet-Broadcasts zur Kommunikation einsetzt, somit Verbindungslos ist und keinerlei Form von Authentifizierung verlangt, kann ein Angreifer jede MAC-Adresse als Antwort auf die Frage nach jeder beliebigen IP-Adresse senden. Dies setzt natürlich die selbe Layer-2 Broadcast-Domain und das fehlen entsprechender Schutzmaßnahmen voraus. Gelingt das Eintragen einer gefälschten MAC-Adresse, werden alle Pakete die an die IP-Adresse des Ziels gesendet werden an die Maschine mit der gewählte MAC-Adresse geschickt. Wird nun die Hardware-Adresse des default Gateway gegen die des Angreifers getauscht, ist es diesem Problemlos möglich den gesamten Netzwerkverkehr außerhalb des lokalen Subnetzes mitzuschneiden und zu verändern.  

Eine weitere Möglichkeit sich als gefälschter Gateway zu positionieren ist es das \textit{Dymamic Host Configuration Protocol} (DHCP) mittels \textit{DPHC-Spoofing} auszunutzen. Bei diesem Angriff wird ein eigener DHCP-Server im Subnet des Opfers positioniert. Dieser gibt zwar gültige IP-Adressen aus, verändert aber die default Gateway und/oder die DNS Server Einstellungen. Somit kann ein Host des Angreifers als Gateway oder DNS-Resolver zwischengeschalten werden und den Netzwerkverkehr kontrollieren. Da DHCP wie ARP auf Ethernet-Broadcasts ohne Authentifizierung setzt sind die Schwachstellen sehr ähnlich. Da der attackierte Client jedoch auch das Annahmen der gefälschten Sitzung über Broadcast verteilt ist das Erkennen etwas einfachen als vom ARP-Spoofing. 

Eine spezielle Methode zum umlenken der Netzwerkpakete stellt das ein Angriff auf das \textit{Internet Control Message Protocol} (ICMP) dar. Dieses Protokoll dient als Grundlage für verschiedenste Unterstützungsaufgaben in IP-Netzen. Am bekanntesten ist die ICMP-Ping Funktion, mit der sich die Konnektivität von Hosts prüfen lässt. Obwohl ICMP oft auf diese Funktionalität reduziert wird beinhaltet es eine Vielzahl von teils wenig bekannten Aufgaben. Die, in Hinblick auf Sniffing, relevante Schwachstelle verbirgt sich dabei in der ICMP-Redirect Funktion, welche für Optimierung des Routing-Prozesses vorgesehen ist. 

Außerdem wird DNS hauptsächlich über UDP verwendet, was ein fälschen (spoofing) der Absenderadresse trivial macht. Da dadurch auch die Transaktions ID und das Absendeport bekannt wird, kann der Angreifer sofort ein gefälschtes Antwortpacket senden und somit den Eintrag der abgefragten Domäne im Resolver des Ziels beliebig verändern oder erweitern. Durch das setzen von hohen TTL Werten kann dieser Zustand lange über die physische Präsenz des Angreifers hinaus aufrecht erhalten werden.

\todo[inline]{Sniffing und Spoofing klar auftrennen?!}

\subsection{DNS Rebinding}

\todo[inline]{DNS Rebinding kurz beschrieben (weil schon Angesprochen)}

\section{Indirekte Angriffe}

\subsection{Reliance Upon Transitive Trust}

\todo[inline]{Angriff über Trust Reliance beschreiben}

\begin{draft}
\begin{markdown}
* Unbemerkte übernehme von Domänen möglich
* Kompromittierung von Stakeholder-Diensten möglich
  * Bei Websites  (HTTP od. uralt Browser mit Mixed Active Content) führt die Kompromittierung eines einzigen Ressourcenservers zum Kompromittierung der gesamten Seite: XSS wird einfach möglich wenn z.B. eine JS-Datei eines Werbeanbieters in die Seite geladen werden kann.
  * Wenn eine einzige aktive Ressource über http nachgeladen wird oder für TLS Attacken (Poodle, ) anfällig ist, kann die Seite und somit der Client angegriffen werden.
* Bei nicht verschlüsselten Netzwerkprotokollen (plain SMTP/POP3/IMAP, FTP, MQTT) kann die Verbindung vollständig übernommen werden.
* In jedem Fall sorgt eine erfolgreiche Attacke zum Übergang der Verfügbarkeitskontrolle an den Angreifer (bis zum Erkennen das Problems und entfernen der eingeschläuschten Einträge)
\end{markdown}
\end{draft}

\subsection{Name Collisions and Leaked Queries} 

\todo[inline]{Leaked Queries wirklich relevant? Vielleicht für Quellen Authentizität?}

\begin{draft}
  \begin{markdown}
* Durch eigene interne TLDs (z.B. .local) kann es zu Kollisionen im globalen Namespace kommen (new TLDs).
* Mögliche Kollisionen können bewusst ausgenutzt werden.
* Durch falsch/schlecht konfigurierte DNS-Resolver können interne Anfragen zu externen DNS-Servern getragen rden -> Information Disclousure
* Speziell bei "home-use" und ohne "LockDown" kann durch lokale Proxies und Resolver von "leakage" betroffen sein. Auch BYOD-Geräte speziell gefährdet wenn durch falsche Konfiguration DNS-Anfragen zum Auflösen internen Ressourcen an externe DNS-Server gestellt werden.
\end{markdown}
\end{draft}

\subsection{C\&C/Exfiltration über DNS Tunneling}

\todo[inline]{Überlegen ob DNS Tunneling wirklich reinpasst}

\begin{draft}
  \begin{markdown}
* Mit allen "offenen" resolvern nutzbar
* Bei "best-practice"-Einstellungen des resolvers sehr langsam
* Nicht einfach zu erkennen
* Für sehr kleine Datenmengen durchaus zuverlässig (C\&C)
* KillChain: Data Exfiltration, Controll
\end{markdown}
\end{draft}

\subsection{DoS-Amplification Angriff}
\label{sec:attack-dosamp}

\todo[inline]{DoS Angriff schreiben}

\begin{draft}
Durch DoS von Resolvern oder DNS-Servern können kritischen Unternehmensdienste kurzfristig außer Betrieb genommen werden. Da die Verbindung zwischen diesen hoch vernetzten Diensten stark von DNS abhängt hat der Ausfall eines einzigen zentralen Dienstes (DynDns Vorfall) schwerwiegende Auswirkungen auf alle anhängenden Dienste.
\end{draft}

\todo[inline]{DoS Angriff schreiben}

\subsection{BitSquatting}

Eine weiter Angriffsmöglichkeit bietet das sogenannte \textit{BitSquatting}. Bei diesem Angriff werden zufällige Fehler im Speicher von Geräten ohne fehlererkennenden Speichermodulen ausgenützt. Es konnte gezeigt werden, dass es durch solche Fehler zum stellen fehlerhafter Anfragen an potenziell existente Domänen kommen kann\cite{Dinaburg2011}. Dieser Effekt kann nun bewusst ausgenutzt werden indem eine große Anzahl an Domänen registriert werden, deren Name sich nur um ein Bit von viel besichten Domänen unterscheidet (siehe Listing \ref{lst:bitquatting}). Da es dadurch zu falschen Anfragen kommt, gibt es auch keine Möglichkeit sich auf Protokoll- bzw. System-Ebene zu schützen. Die einzige effektive Lösung stellt der Einsatz von \textit{Error-correcting code memory} Hardware dar.  

\begin{lstlisting}[caption={Drei mögliche BitSquatting-Domänen für die Zieldomäne \texttt{amazon.com}}, label={lst:bitquatting}]
amazon.com = 61   6d   61 7a   6f   6e 2e 63 6f 6d
           ... 01101101 ... 01101111 ... 
aeazon.com = 61   65   61 7a   6f   6e 2e 63 6f 6d  
           ... 01100101 ...
                   ^
a-azon.com = 61   e2   61 7a   6f   6e 2e 63 6f 6d 
           ... 00101101 ...
                ^
amazgn.com = 61   6d   61 7a   67   6e 2e 63 6f 6d
                        ... 01100111 ...
                                ^
\end{lstlisting}

\section{Übersicht}
\label{sec:Attacks-Summary}

\todo[inline]{Angiffszusammenfassung schreiben. (Nicht sicher ob notwendig)}

\chapter{Empfehlungen}
\label{chap:solutions}
Aufgrund der in den vorherigen Kapiteln beschriebenen Problemen und Angriffen können insgesamt 6 Lösungsvorschläge erstellt werden. Diese als Empfehlungen zu verstehenden, allgemeinen Beschreibungen werden bereits von verschiedenen Technologien erfüllt. Auf diese wird im diesem folgenden Kapitel \ref{chap:technologies} genauer eingegangen.

\section{Authentizität der Records}
\label{sec:solution-recordauth}
Wie in Kapitel \ref{sec:thread-auth} beschrieben wurde, ist die Authentizität der RR ein zentrales Problem in DNS. Zur Lösung wurde schon früh der Einsatz kryptografischer Signaturen postuliert. Mithilfe dieser könnten alle Einträge bei der Erstellung signiert, zusammen mit den Signaturen übertragen und dann von der Gegenstelle validiert werden. Dadurch wird es möglich die Integrität und Authentizität der empfangenen Einträge eindeutig sicherzustellen. Werden die Zonendateien nur bei Änderung signiert und der private Schlüsselteil anschließend wieder entfernt, besteht sogar die Möglichkeit Server-Konfigurationen vor ungewollter Veränderung zu schützen. 
Die zentralen Fragen sind dabei, wie bei allen Signaturverfahren, das Schlüsselmanagement und die damit verbundene Vertrauensstellung zwischen den einzelnen Parteien. Das notwendige Vertrauen kann dabei auf verschiedenste Wege erreicht werden (siehe Kapitel \ref{chap:itsecurity}).
Des weitern ist zu erwähnen, dass die Integrität der Nachrichten trotz Signatur nicht immer garantiert werden kann. Einige Signaturmechanismen, konkret die darin genützten Hash-Algorithmen, weisen bekannte Schwächen auf, die bei einem Angriff ausgenützt werden können. Ein aktuelles Beispiel stellt der Angriff auf den weit verbreiteten SHA-1 Algorithmus dar\cite{Stevens2017}. 

\section{Authentizität der autoritativen Server}
Neben der in Abschnitt \ref{sec:solution-recordauth} erwähnten Authentizität der RR selbst, spielt auch die Authentizität der autoritativen DNS-Server selbst eine Rolle. Einige der angeführten Angriffe, nutzen den Umstand, dass Resolver keine Möglichkeit besitzt die Identität der Gegenstellen zu prüfen. Dies ist, aufgrund des zustandslosen Grundkonzepts, selbst zwischen Anfrage und Antwort zutreffend. Nach Standard verlässt sich ein Resolver ausschließlich auf die Kombination aus Adresse, Port und TXID um eine Antwort einer ausstehenden Anfrage zuzuordnen. Dies ist Grundlage für die meisten Angriffe gegen Resolver.

Eine Lösung besteht prinzipiell in der Signatur der Records, ist jedoch auch einfacher möglich. Lässt man die \textit{Authentizität der Records} außer acht, genügt es eine kurzlebige Vertrauensstellung zwischen Resolver und autoritativen Server herzustellen. Dies kann über klassischen hybriden Verschlüsselungsverfahren erreicht werden, die asymmetrische Kryptografie für die Authentifizierung und den Schlüsselaustausch, sowie symmetrische Kryptografie für die Datenverschlüsselung einsetzen.

\section{Authentizität des Resolvers}
In Kapitel \ref{sec:thread-auth} wurde erwähnt, dass es in DNS-Netzwerkprotokoll keine Möglichkeit gibt die Authentizität der Gegenstelle, im speziellen des Resolvers, zu prüfen. Abhilfe für dieses Problem schafft einerseits das zuvor besprochene kryptographische Signieren der Einträge, da so bei einem Eingriff keine unbemerkte Modifikation des Inhalts möglich ist. Dieser Schutz scheitert jedoch an der praktischen Umsetzung, da es einen Fallback-Mechanismus ausschließen müsste, da ansonsten eine Downgrade-Attacke möglich wäre\cite{Bau2010}. Bei diesem Angriff werden dafür die Signaturen aus dem Antwort-Paket entfernt. Dies erzeugt den Eindruck , dass der Server die Unterstützung von DNSSEC eingestellt hat. Die Anfrage werden in anfälligen Konfigurationen nach ungeschütztem DNS abgearbeitet und sind damit wieder normal angreifbar.  

Eine mögliche Lösung dieses Problem stellt das einführen eines Authentifizierungsverfahren dar. In den meisten Fällen wird dafür eine asymmetrischer Kryptografie basierende Methode eingesetzt. Dieses besteht aus 2 Schritten: Als erstes teilt die fragliche Endstelle seinen öffentlichen Schlüssel mit, der Client trifft nun eine Entscheidung ob er dem Schlüssel vertraut oder nicht. Wir schon in Abschnitt \ref{sec:solution-recordauth} erwähnt, gibt es dafür verschiedene Methoden um das notwendige Vertrauen zu dem gegebenen Schlüssel herzustellen und zu prüfen. Der zweite Schritt stellt sicher, dass die Endstelle auch im Besitz des passenden, privaten Schlüssels ist. Dabei wird oft die Authentifizierung und die Sitzungsschlüsselübertragung zusammengefasst. Der Client überträgt dabei seinen Informationsteil für die Sitzungsschüsselerstellung verschlüsselt mit dem öffentlichen Schlüssel des Gegenübers. Eine erfolgreiche Aushandlung des Sitzungsschlüssels ist damit nur möglich wenn die Endstelle die Nachricht auch entschlüsseln kann, was nur mit passendem privaten Schlüssel möglich sein sollte. 

\section{Vertraulichkeit der Verbindung}
Wie in Abschnitt \ref{sec:thread-priv} dargelegt ist DNS-Privacy zu einem wichtigen Thema geworden. Die nachhaltige Lösung zur Sicherstellung der Vertraulichkeit ist, wie damals auch bei HTTP, der Einsatz einer geeigneten Transportverschlüsselungstechnologie. Auf welcher Ebene diese Stattfindet ist dafür grundsätzlich nicht relevant. Der Einsatz von etablierten Technologien wie TLS liegt jedoch nahe, obwohl für DNS auch alternative Lösungen entwickelt wurden. 

\section{Trennen von Adresse und Anfrage}
\label{sec:goals-sourceanon}
Um die sensiblen Verbindungsdaten zu Schützen ist es wichtig die Anonymität der anfragenden Person zu wahren. Konkreter sollen Adressen die mit der entschlüsselten Anfrage in Verbindung steht, nicht auf die anfragende Person zurückführbar sein. Durch den Einsatz von rekursiven Resolvern wird diese Kombination aus Adresse und Anfrage zwar vor den autoritativen DNS-Servern verborgen, verschiebt das Problem jedoch lediglich auf den Resolver. 

Die Lösung besteht nun in der simplen Trennung von Adresse des Clients und dessen entschlüsselter Anfrage bzw. Antwort. Um dies zu ermöglichen ist es notwendig, das der erste, vom Client direkt angesprochene Server, nicht in der Lage ist die echte DNS-Anfrage oder Antwort zu entschlüsseln. Die Aufgabe dieses Servers besteht darin die wahre Adresse des Clients zu verbergen und an die Anfrage von einen anderen Adresse aus an einen Resolver weiterzuleiten. Der Resolver muss dann in der Lage sein, die Anfrage zu Entschlüsseln, ist aber nicht mehr in Kenntnis über die echte Adresse des Fragestellers. Damit ist das problematische Tupel aus Adresse und Daten aufgelöst, was zur Verbesserung der Privacy beiträgt \cite{Schmitt2018}.

\section{Verbindungbehaftete Protokolle einsetzen}
\label{sec:goals-trafficamp}
Abgesehen von der Verfügbarkeit des DNS selbst, wird es, wie in Abschnitt \ref{sec:thread-dosamp} und \ref{sec:attack-dosamp} beschrieben, auch selbst als Verstärker (Traffic Amplification; DoS Amplification Attack) für DoS-Attacken eingesetzt. Es ist daher für zum Schutz anderer Internet-Services essenziell, dass DNS und dessen Implementierungen nicht lohnender für Traffic-Amplification Versuche sind als andere Services.

Die einfachste Möglichkeit einer Lösung besteht im Ablösen des Verbindungslosen UDP zugunsten eines Verbindungsorientierten Protokolls wie TCP. Der notwendige Verbindungsaufbau könnte im Falle eines klassischen IP-Spoofings nicht abgeschlossen werden, was das Stellen einer Anfrage verhindern würde. Damit wäre DNS nicht mehr als Verstärker nutzbar. Dieser Vorteil ist jedoch direkt mit dem Nachteil verbunden, da der Verbindungsaufbau viel Zeit in Anspruch nimmt und zusätzliche Ressourcen auf dem Server bindet. Wie festgestellten werden konnte, ist die Auswirkung auf die Performance in den meisten Fällen jedoch stark überschätzt\cite{Zhu2015}.

Neben TCP können auch Protokolle wie DTLS, trotz UDP Transportprotokoll, eingesetzt werden. DTLS baut eine Sitzung auf, was die Möglichkeit zum Einsatz als Verstärker ebenfalls erschweren würde, da der Server erst nach erfolgreichem Handshake Anfragen akzeptiert\cite{rfc6347}. 


\chapter{Technische Umsetzungen}
\label{chap:technologies}
% Kurzbeschreibung, Löst welches Problem?, Vorteile/Nachteile, Verbreitung
% Mehr Struktur? Bilder?!

\section{DNSSEC}

DNSSEC, als eine Sammlung an IETF-Spezifikationen, erweitert DNS mit dem Ziel die Integrität und Authentizität der Ressourceneinträge sicherzustellen (\cite{Arends2005}). Dies wird über die krypographische Signatur der Einträge erreicht. Um die Signaturen validieren zu können, wird eine Chain-Of-Trust genützt, die ihren Ankerpunkt im von der ICANN veröffentlichten DNSSEC-Root-Key besitzt. Der Umstand, dass die ICANN somit als \textit{single point of trust} fungiert und somit den Grundgedanken der \textit{Verteiltheit des Internets} unterwandert ist einer der Hauptkritikpunkte an DNSSEC. Des weiteren findet die Erweiterung aufgrund der langen Entwicklungszeit, der (verglichen zum DNS-Protokoll) hohen Komplexität und dem erhöhtem administrativem Aufwand, nur wenig anklang. Seit der finalen Veröffentlichen 2005 konnten zwar 90 Prozent der TLD Betreibenden zum signieren ihrer Zonen bewegt werden, die Verbreitung signierter 2LDs ist mit ca. 4 Prozent jedoch viel zu gering um Maßgeblich zur Sicherheit des globalen DNS beizutragen (http://rick.eng.br/dnssecstat/). Zusätzlich wurde das Schutzziel der Vertraulichkeit bewusst bei der Konzeptionierung ausgeklammert. Aufgrund verschiedenster Entwicklungen der nahen Vergangenheit ((Türkei, Deutschland, USA Prism, etc)) ist der Bedarf nach einer vertrauten Möglichkeit zur Auflösung von Namen im Internet stark angestiegen. Betrachtet man nun zusätzlich den Umstand, dass das DNSSEC Netzwerkprotokoll noch stärker als reines DNS, für DoS-Amplification genützt werden kann, ist klar warum diese Spezifikationssammlung so wenig positive Resonanz erhält.

\todo[inline]{Anmerkung Einfügen: Problem mit Privacy ist, dass der ISP oder der Staat auch einfach SNI oder IP Kommunikation untersuchen kann. Damit lassen sich Verkehrsdaten ähnlicher Qualtität bilden. Privicy ist also nicht das beste Argument. (Referenz? Einfach weglassen?)}

\todo[inline]{In Conclusio Hinweis auf die Tatsache, dass die CoT zur Zeit die einzige, einfache (weil DNSCrypt auch) darstellt, um die Authentizität gut zu prüfen.}

\section{DNSCurve}

Aufgrund der schlechten Akzeptanz und dem fehlenden Schutz der Vertraulichkeit veröffentlichte der US-amerikanische Kryptograph Daniel J. Bernstein (auch bekannt als DJB) eine eigene Lösung namens DNSCurve. Diese dient zur Sicherstellung einer vertraulichen und authentischen Kommunikation zwischen rekursiv auflösendem Server und den authentitiven Servers. Sie baut auf dem, von DJB entwickelten, elliptische Kurven-Kryptosystem Curve25519 auf und verwendet so, im vergleich zu DNSSEC, welches RSA einsetzt, elliptic curve cryptography (ECC), für asymmetrische Operationen. Außerdem wird der, ebenfalls selbst entwickelte, Poly1305 message authentication code (MAC) für die Verschlüsselung und Echtheitsprüfung der Nachrichten eingesetzt. Der Einsatz von ECC sorgt, im Vergleich zu RSA, für eine signifikante Verbesserung der Performance bei der Schlüsselaushandlung, da weit kürzerer Schüssel bei gleichbleibender Sicherheit eingesetzt werden können\cite{Gupta2002}. Der in RFC7905 spezifizierten Poly1305 Algorithmus ist ebenfalls auf Geschwindigkeit, unter Einhaltung der Sicherheitsansprüche, optimiert\cite{Bernstein2005}. Somit war DNSCurve, bei dessen Veröffentlichung 2009, DNSSEC, speziell im Bereich Performance, weit überlegen. Da der höhere Leistungsansprüche der Validierung bis Heute eine starkes Argument gegen den Einsatz von DNSSEC dargestellt, wurde DNSCurve, trotz des ungewöhlichen Algorithmen, durchaus positiv aufgenommen \cite{Henry2013}. Dies Überrascht, da DNSCurve, wie D. Kaminsky feststellt, ein konzeptionellen Problem im Bereich des \textit{Trust Establishment} Prozesses ausweißt \cite{Kaminsky2011}. Das Protokoll verlässt sich beim Auffinden der öffentlichen Schüssel auf einen speziellen NS-Eintrag der zu Ziel-DNS-Domäne. Da DNSCurve ohne Erweiterungen des DNS-Protokolls auskommt, wird dieser Eintrag alleinig durch dessen spezielles Format aufgefunden. Darüber hinaus wird der Schlüssel in den NS-Eintrag selbst kodiert. Ein Angreifer in aktiven MitM Position, könnte somit, durch simples Umschreiben des Eintrags, eine Kommunikation über DNSCrypt unterbinden. Selbst wenn kein Fallback zur Klartextkommunikation erfolgt, besteht noch immer die Möglichkeit, ein eigenes Schlüsselpaar zu generieren und den öffentlichen Schüssel des Ziels gegen einen eigenen zu tauschen. In der Spezifikation scheint zwar die Möglichkeit zum Aufbau einer CoT auf, da eine zentrale Vertrauensinstanz bewusst fehlt, wird diese jedoch von Kaminsky als \textit{ineffektiv} bezeichnet, da sie laut ihm keine zufriedenstellende Lösung des Problems \textit{Key Management} birgt. Es ist somit fraglich, ob DNSCurve in der Lage ist die Vertraulichkeit und Authentizität von DNS sicherzustellen.  

\section{DNSCrypt}

Zusätzlich zu DNSCurve und DNSSEC wurde 2011 das Netzwerkprotokoll DNSCrypt entwickelt. Dieses ähnelt DNSCurve im grundlegenden Aufbau insofern es die von DJB entwickelten Algorithmen Curve25519 und Poly1305 zum Schlüsselaustausch und zur Überprüfung der Nachrichten verwendet und das DNS Netzwerkprotokoll ohne weiter Modifikationen für die Kommunikation einsetzt. Der größte Unterschied liegt darin, dass DNSCrypt die kommunikation zwischen Resolver und rekursivem DNS Server schützt. Um dies zu erreichen, werden kurzlebige Zertifikate eingesetzt, welche zur Verschlüsselung der Anfragen und Validierung der Antworten eingesetzt werden. Diese Zertifikate sind mit einem langlebigen Schüssel signiert, wobei der öffentliche Schüssel von jedem Client explizit in eine Liste an vertrauenswürdigen Schüsseln aufgenommen werden muss.\cite{Denis2016} Dies umgeht zwar die konzeptionelle Schwäche von DNSCrypt formal, ist jedoch ohne entsprechende Unterstützungsmethoden unpraktikabel. Eine in manchen Implementierungen verwendete Lösung nach dem \textit{Trust-on-first-use}-Prinzip (TOFU) wird als nicht zuverlässig erachtet.\cite{Wendlandt2008} Somit stellt DNSCrypt, ähnlich wie DNSCurve, zwar eine sichere  Möglichkeit zum Übertragen von DNS-Anfragen und Antworten an den auflösenden DNS Server dar, vernachlässigt jedoch ebenfalls das zentrale Thema des \textit{Schlüsselmanagements}.    

\section{DNS-over-TLS}

Das Kernproblem der DNS Security besteht in der fehlender Authentizität und Vertraulichkeit der Übertragung. Diesen Umstand hat DNS mit vielen älteren Netzwerkprotokollen gemein, speziell im Internetumfeld ist HTTP, neben DNS, eines der verbreitetsten Protokolle. Als Lösung für HTTP wurde SSL bzw. TLS festgelegt, welches, nach seiner Spezifizierung 2000, das meist genützte Transportverschlüsselungsprotokoll der TCP/IP-Suite wurde. Es liegt also die Sicherheitsschwäche von DNS ebenfalls mit TLS als Transportprotokoll zu beheben. Das Problem besteht dabei in der ursprünglichen Intention DNS Verbindungslos auszulegen, ob das Protokoll so simpel und effizient wie möglich zu gestalten. Da TLS ein verbindungsorientiertes Protokoll (meistens TCP) als Trägerprotokoll verlangt, stehen diese Anforderungen im Widerspruch zueinander. Der Zeitaufwand für das Aufbauen einer Verbindung, zusammen mit dem erhöhten Ressourcenaufwand auf der Serverseite wurde lange Zeit als finales Gegenargument gegen den Einsatz von DNS über TCP (DNS-T) verwendet. Wie jedoch Zhu et at. 2015 feststellen konnte, ist dieses Argument für moderne Systeme nicht mehr zulässig \cite{Zhu2015}. Das anbieten von DNS über TLS (DNS-over-TLS; DoT) wurde in den IETF Dokumenten RFC7858\cite{Hu2016} und RFC8310\cite{Dickinson2018} spezifiziert und ist daher sein 2016 als sichere Methode zur Übertragung von DNS Anfragen und Antworten zwischen 2 Endpunkten zu betrachten. Die Aufgabe der Quellenauthentizitätskontrolle wurde auf das TLS Protokoll übertragen und folgt damit den selben Mechanismen die schon von HTTPS bekannt sind. Die über ein X509 Zertifikat authentisierende Gegenstelle wird im Zuge des TLS-Handshake über vorinstallierte Stammzertifikate von Zertifizierungsstellen authentifiziert. Die Vertrauensstellung zu den Zielservern ist daher implizit transitiv über das Vertrauen zur ausstellenden Stelle des Serverzertifikats hergestellt.        

\section{DNS-over-HTTPS}

Einen mit DoT vergleichbaren Ansatz wählt die im IETF Draft \textit{DNS Queries over HTTPS} (DoH) beschriebene Technologie \cite{Mcmanus2018}. Diese Technik nützt HTTPS statt TLS als Trägerprotokoll und ermöglicht damit auch nativen Web Applikation die Auflösung von DNS Anfragen. Der Entwurf sieht zusätzlich die Möglichkeit einer \textit{Server Push} Funktion vor, welche es DoH-DNS-Servers erlaubt, von sich aus Pakete an Clients zu senden. Es wird argumentiert, dass so der Auflösungsprozess beschleunigt werden kann, da zusätzlich zur eigentlichen Anfrage, zugehörige, andere Einträge mitgesendet werden können. Diese müssen dann nicht nochmals angefragt werden, sondern sind schon im Cache des Resolvers geladen. Zusätzlich soll es möglich sein, die bestehende HTTP-Caching-Infrastruktur für DoH-Anfragen mitzubenützen. Die Funktionsweise im Hinblick auf Vertraulichkeit und Authentizität unterscheidet sich nicht von DoT und ist somit grundlegend valide.

\section{QNAME minimisation}

Die grundlegende Idee hinter QNAME minimisation ist es, die Informationen der Anfragen an authoritative DNS-Server zu minimieren. Recursive Resolver senden ohne entsprechende Einstellung immer den vollen Domänen-Namen in ihren Anfragen, selbst wenn zu diesem Zeitpunkt klar ist, dass der Zielserver (z.B. ein Root-DNS-Server) keine vollständgige Antwort liefern werden kann. Um die Privicy von Client eines Recursive-Resolvers zu verbessern kann nun QNAME minimisation eingesetzt werden um nur noch den Teil der Domäne abzufragen, für den der Zielserver auch authoritativ ist. Dies führt dazu, dass die DNS Auflösung beteiligten DNS-Server nicht mehr auf die gewünschte Zieldomäne schließen können, solange sind nicht für diese verantwortlich sind. Diese Technologie ist in RFC7816 spezifiziert und wird von aktueller open-source DNS-Server-Software unterstützt. Da sich diese Technik auf den rekursiven Auflöseprozess auswirkt, hat sie keine Auswirkungen auf Stub- oder Forwarding-Resolver.

\section{EncDNS und Oblivious DNS}

Wie in Abschnitt \ref{sec:solutions-PrivRequest} angesprochen, besteht aus Sicht der Vertaulichkeit das Problem, dass der gewählte Recursive-Resolver einfach Zugriff auf die Kombination von Client-Adresse und angefragte Zieladresse besitzt. Spezielle relevant erhält das Thema, da Trusted Resolver aktuell einen starken Zulauf erhalten und so immer mehr dieser Daten binden. Lässt man das komplette Ersetzen von DNS außer acht, besteht nur noch die Möglichkeit diese Verknüpfung aufzuheben. 
EncDNS\cite{Herrmann2014} und Oblivious DNS\cite{Schmitt2018} sind beides ähnliche Technologien mit eben diesem Ziel. Um die Verschleierung der Client-Adresse zu erreichen wird ein jeweils eigener Stub-Resolver eingesetzt. Dieser erstellt spezielle, DNS Standardkonforme Anfragen mit einem speziellen Aufbau. Die eigentliche DNS-Anfrage wird dabei verschlüsselt, als Präfix zu einer speziellen Domäne verpackt. Der authoritative Server dieser Domäne muss entsprechend fähig sein das Format zu verstehen. Dieser Aufbau veranlasst den Recursive-Resolver die Anfrage normal abzuarbeiten, was diese schlussendlich zum authoritativen Server leitet. Dieser entschlüsselt die spezielle Anfrage und führt die eigentliche Auflösung durch. Danach wird die verschlüsselte Antwort formuliert und an den Resolver zurückgeschickt, welcher sie an den Client weitergibt.
Wie man sieht, ist der Recursive-Resolver durch die Verschlüsselung nicht möglich die echte Anfrage und Antwort zu lesen. Des weiteren ist aber auch der authoritative DNS-Server nicht in Kenntnis über die echte Adresse des Clients. Somit wurde die Verknüpfung zwischen Client-Adresse und Anfrage gelöst.
Der unterschied wirschen EncDNS und ODNS besteht nun in den eingesetzten Protokollen zur Verschlüsselung und zum Transport der Nachrichten. EncDNS ist stark von DNSCurve inspiriert und baut somit auf die selben Algorithmen und Formaten auf. ODNS folgt zwar dem gundlegenden Konzept von EncDNS, legt den Fokus jedoch auf Performance und Verfügbarkeit. Diese Entscheidung spiegelt sich vor allem im Einsatz eines verteilten Key Distribution Mechanismus wieder. Dieser erlaubt es dem System einen, für den Resolver, optimalen DNS-Server zu wählen ohne das selbe Schlüsselpaar auf alle DNS-Server verteilen zu müssen. Als Algorithem setzt ODNS AES fpr Sitzungsschlüssel und ECIES für den Schlüsseltransport ein.
Wie schon im Abschnitt DNSCurve erwähnt, erben damit beide Systeme sie selbe Schwäche im Trust-establishment Prozess. Es wäre einem Angreifer, der den kompletten Netzwerkverkehr des Opfers unter kontrolle hat, möglich den initialen öffentlichen Schlüssel zu verändern und sich somit als Resolver und EncDNS bzw. ODNS Server zu positionieren.

\section{Alternative Ansätze: Namecoin, GNS, RAINS}
\todo[inline]{write (low priv!) Alternative Ansätze: Namecoin, GNS, RAINS}

\section{Übersicht}
In Tabelle \ref{tbl:technologiesSummery} sind alle beschriebenen Technologien anhand ihrer Schutzziele aufgereiht.

\footnotetext[1]{\label{fn1}Nicht Sicherheitsrelevant, da kein Verfälschen der Antwort möglich}
\footnotetext[2]{\label{fn2}Minimal mögliche Informationsweitergabe an authentitive DNS-Server}
\footnotetext[3]{\label{fn3}Es bestehen Zweifel am Key Management Konzept was die praktische Authentizitätssicherheit fraglich macht}

\begin{table}[!ht]
\caption{Technologien anhand ihrer Schutzziele. Gegliedert nach Integrität (I), Authentizität (A) und Vertraulichkeit (V).} \label{tbl:technologiesSummery} \centering \small
\begin{tabular}{r|c|c|c|c|c|c|}
\cline{2-7}
\multicolumn{1}{l|}{} & 
\multicolumn{1}{c|}{I} & 
\multicolumn{2}{c|}{A} & 
\multicolumn{3}{c|}{V} 
\\

\cline{2-7} 
\multicolumn{1}{l|}{} & 
\rotatebox[origin=c]{90}{\begin{tabular}[c]{c}Nachrichten \end{tabular}} & 
\rotatebox[origin=c]{90}{\begin{tabular}[c]{c}Resolver \end{tabular}} & 
\rotatebox[origin=c]{90}{\begin{tabular}[c]{c}Antwort \end{tabular}} & 
\rotatebox[origin=c]{90}{\begin{tabular}[c]{c}Verbindung \end{tabular}} & 
\rotatebox[origin=c]{90}{\begin{tabular}[c]{c}Anfrage am Resolver \end{tabular}} & 
\rotatebox[origin=c]{90}{\begin{tabular}[c]{c}Anfrage am auth. DNS \end{tabular}} 
\\ 

\hline
\multicolumn{1}{|r|}{DNS}           & 
N                                   & % I d. Nachricht
N                                   & % A d. Resolvers
N                                   & % A d. Antwort
N                                   & % V d. Verbindung
N                                   & % V d. Resolver Antwort
N                                     % V d. auth. DNS                          
\\ 

\hline
\multicolumn{1}{|r|}{DNSSEC}        & 
Y                                   & % I d. Nachricht
N\textsuperscript{\ref{fn1}}        & % A d. Resolvers
Y                                   & % A d. Antwort
N                                   & % V d. Verbindung
N                                   & % V d. Resolver Antwort
N                                     % V d. auth. DNS       
\\ 
\hline
\multicolumn{1}{|r|}{DNSCurve\textsuperscript{\ref{fn3}}}      & 
Y                                   & % I d. Nachricht
N\textsuperscript{\ref{fn1}}        & % A d. Resolvers
Y                                   & % A d. Antwort
Y                                   & % V d. Verbindung
Y                                   & % V d. Resolver Antwort
N                                     % V d. auth. DNS        
\\ 

\hline
\multicolumn{1}{|r|}{DNSCrypt}      & 
Y                                   & % I d. Nachricht
Y                                   & % A d. Resolvers
N                                   & % A d. Antwort
Y                                   & % V d. Verbindung
N                                   & % V d. Resolver Antwort
N                                     % V d. auth. DNS                    
\\ 

\hline
\multicolumn{1}{|r|}{DoT / DoH}     & 
Y                                   & % I d. Nachricht 
Y                                   & % A d. Resolvers 
N                                   & % A d. Antwort 
Y                                   & % V d. Verbindung 
N                                   & % V d. Resolver Antwort 
N                                     % V d. auth. DNS                       
\\ 

\hline
\multicolumn{1}{|r|}{QNAME min.}    & 
N                                   & % I d. Nachricht 
N                                   & % A d. Resolvers 
N                                   & % A d. Antwort 
N                                   & % V d. Verbindung 
N                                   & % V d. Resolver Antwort
Y\textsuperscript{\ref{fn2}}          % V d. auth. DNS                       
\\ 

\hline
\multicolumn{1}{|r|}{EncDNS\textsuperscript{\ref{fn3}} / ODNS\textsuperscript{\ref{fn3}}} & 
Y                                   & % I d. Nachricht 
Y                                   & % A d. Resolvers 
N                                   & % A d. Antwort 
Y                                   & % V d. Verbindung 
Y                                   & % V d. Resolver Antwort
N                                     % V d. auth. DNS       
\\ 

\hline
\end{tabular}
\end{table}

\todo[inline]{noch Tabelle mit Fokus Sicherheitszeilen einfügen}

\chapter{Implementierung}
\label{chap:implementation}

Um die in Kapitel \ref{chap:threads} beschriebenen Probleme zu mindern oder gar zu beheben, kann eine Auswahl an vorgestellten Technologien (siehe \ref{chap:technologies}) genutzt werden. Das Ziel ist dabei der Schutz einer kleinen Anzahl von Clients durch den Einsatz eines speziell konfigurierten lokalen, rekursiven Resolvers. Es ist dabei nicht sinnvoll alle anwendbaren Techniken einzusetzen, da jede mit Kosten in Performance einhergeht, wobei die Vorteile mancher Kombinationen redundant sind.

\section{Position und Resolver Typ}
Die möglichen Positionen und Betriebsarten eines Resolvers (siehe Abb. \ref{img:impl-resolverpositions}) wurden bereits im Abschnitt \ref{sec:dnsresolverposition} behandelt. Für die Auswahl wurden drei Kriterien betrachtet: Angreifbarkeit, Performance und Privacy.
\begin{figure}[hb]
    \centering
    \includegraphics[width=\textwidth,trim={8mm 8mm 8mm 8mm},clip]{Impl_ResolverPositions}
    \caption{Stellt die möglichen Positionen eines Resolvers, aus Sicht des Clients, dar.}
    \label{img:impl-resolverpositions}
\end{figure}

\paragraph{Local-Loopback Resolver (1)}
Diese Position ist bei einem, auf die Verbindung zwischen Recursive Resolver und Stub-Resolver abgezielten, Angriff die sicherste. Werden Techniken wie DoT, DoH oder DNSCrypt im Forwarding Modus eingesetzt, kann darüber hinaus die Vertraulichkeit und Authentizität bis zum Recursive Resolver sichergestellt werden. Die Nachteile ist die fehlende Möglichkeit eines geteilten Caches, die erhöhte Ressourcenverbrauch am Client und der hohe Wartungsaufwand. Außerdem können keine Internet-Of-Things (IoT) Geräte bedient werden, da bei diesen eine Installation eines lokal laufenden Resolvers nicht möglich ist.

\paragraph{Local Network Resolver (2)}
Durch einen Resolver im Lokalen Netzwerk, besteht zwar die Gefahr von Angriffen auf Netzwerkebene, es wird jedoch der Einsatz eines gemeinsamen Caches möglich und eine hohe Geräte-Kompatibilität stakt verbessert. In kleinen, gut gesicherten Netzwerken kann dieses Angriffszenario aufgrund des geringen Risikos in Kauf genommen werden. Angriffe von außen sind jedoch durchaus möglich und wahrscheinlich, sollte der Resolver nicht speziell geschützt werden.

\paragraph{ISP Resolver (3)}
Da, wie in Abschnitt \ref{sec:thread-priv} beschrieben, nicht auf die Verantwortlichkeit von ISPs oder staatlichen Institutionen vertraut werden sollte, ist das Nutzen eines ISP-Resolvers, vor allem in Hinblick auf Privacy, nicht zu empfehlen. Einzige Ausnahme stellt dabei der Einsatz von Technologien wir DNSCurve, EncDNS und ODNS dar, da diese eine Verschlüsselung der Anfragen durch den Resolver hindurch unterstützen. Aufgrund kurzer Laufwege und großen Caches sind diese Resolver in den meisten Fällen durchaus performant.

\paragraph{Private Resolver (4)}
Betreibt man seinen eigenen Resolver im Internet erhält man dadurch Kontrolle über die Privacy-Einstellungen des Resolvers. Abgesehen davon, setzt man sich damit jedoch allen möglichen Angriffen, nicht nur auf DNS, aus. Darüber hinaus sind die meisten Stub-Resolver nicht dazu in der Lage über sichere Protokolle mit dem Resolver zu kommunizieren. Abgesehen davon muss darauf geachtet werden, dass keine Zuordnung zwischen der Resolver-IP in einzelnen Personen hergestellt werden kann. Da diese Nachteile den Vorteilen stark überwiegen kann vom Einsatz eines eigener, privater Resolvers abgeraten werden.

\paragraph{Public Resolver (5)}
Der direkte Einsatz eines Trusted Public Resolvers hat viele Vorteile. Die Performance ist aufgrund guter Anbindungen, eines professionellen Betriebs und großen Caches meist gut. Viele Resolver haben spezielle Schutzmaßnahmen gegen Cache Poisoning- oder DoS-Attacken im Einsatz. Außerdem ist die Anonymität gegenüber autoritativen DNS-Servern, aufgrund der hohen Zahl an Users, als ausreichend anzusehen. Der einzige, schwerwiegende Nachteil besteht in der Gefahr von Privacy-Verletzungen durch den Betreiber des Dienstes. Hier ist entweder eine Abwägung zu treffen oder eine Lösung zur vollständigen Anonymisierung zu finden. 

\section{Technologie-Stack}
Zur Auswahl des Sets an Technologien (hier als ``Stack'' bezeichnet) wurde aufgrund der, am Ende des Kapitels \ref{chap:attacks} angeführten, Übersicht erstellt. Dabei wurde darauf geachtet, jede der unter \ref{chap:solutions} abgegebenen Empfehlungen gerecht zu werden. Um die Auswahl besser verstehen zu können, wird hier kurz auf jede einzelne der Techniken eingegangen.

\paragraph{DNSSEC}
DNSSEC stellt aktuell die einzige Möglichkeit zur Echtheitsprüfung von RRs selbst dar (siehe Abschnitt \ref{sec:tec-dnssec}). Das Abfragen und Validieren der DNSSEC RRSig-Records ist somit Pflicht für jeden Resolver mit Sicherheitsfokus. Die Validierung kann an verschiedenen Stellen erfolgen. Da die Validierung aufgrund der kryptographischen Operationen durchaus ressourcenintensiv sein kann, ist es Sinnvoll diese Operation nicht auf jedem Client durchzuführen. Abgesehen davon fehlt vielen Stub-Resolvern die Möglichkeit zur selbstständigen Validierung. Wird ein Forwarding-Resolver eingesetzt, kommt es auf die Konfiguration und Implementierung an ob dieser eine Validierung durchführt oder sich auf den nachgeordneten Recursive Resolver verlässt. Kann dem rekursiven Resolver vertraut werden, ist die Verifizierung durch diesen, aus Effizienzgründen, zu bevorzugen.

\paragraph{DNS-over-TLS}
Wie in Abschnitt \ref{sec:tec-dot} beschrieben stellt DoT einen simplen Aufsatz zum klassischen DNS Netzwerkprotokoll dar. Für den Transport wird TLS über TCP auf Port 853 genutzt\cite{rfc7858}. Damit ist die Verbindung zwischen Resolver und Recursive Resolver geschützt. 

\paragraph{Address-Obfuscation über NAT}
Um auch die Vertraulichkeit der Anfragen erhalten zu können, kann zusätzlich der Zusammenhang zwischen der Client-Adresse und der Anfrage aufgelöst werden. Dies kann, wie schon in Abschnitt \ref{sec:tec-nat} erwähnt, durch den Einsatz eines NAT Servers erreicht werden. Zur Umsetzung kann nahezu jedes modernen Server-Betriebssystem herangezogen werden.

\section{Konzept}
Aus den beschriebenen Vor- und Nachteilen ergibt sich der Einsatz eines lokalen Forward-Resolver als optimaler Kompromiss aus Sicherheit, Kompatibilität und Performance. Der beschriebene Technologie-Stack ist in der Lage alle in Kapitel \ref{chap:solutions} vorgestellten Empfehlungen zu erfüllen. In Abbildung \ref{img:impl-architecture} wird der schematische Ablauf der Kommunikation gezeigt. Es werden dabei zusätzlich die Übergangsstellen der Verschlüsselung und Client-Identifizierbarkeit gezeigt.
\begin{figure}[hb]
    \centering
    \includegraphics[width=\textwidth,trim={5mm 5mm 5mm 5mm},clip]{Impl_Architecture}
    \caption{Darstellung des schematischen Kommunikationsweg des Test-Aufbaus.}
    \label{img:impl-architecture}
\end{figure}

Um die Privacy der lokalen Clients und damit der Nutzenden zu wahren, wurde der Aufbau nach einem einfachen Grundkonzept entworfen: Entschlüsselte DNS-Nachrichten dürfen nur auf vom User direkt kontrollierten Komponenten mit der externen Internet-Adresse verknüpfbar sein. Dadurch wird verhindert, dass Betreiber der Transportnetzwerke oder der nachgeordneten DNS-Komponenten private Daten mit der Identität der Nutzenden verknüpfen.

Des weiteren wird durch DoT jede Form von Sniffing und Spoofing Attacken unmöglich gemacht. Da der Forwarding-Resolver selbst keine normalen DNS Anfragen von außerhalb des lokalen Netzwerks annimmt, ist der Resolver auch gegen DNS DoS Attacken durch externe Angreifer geschützt. Die DNSSEC Validierung erfolgt je nach Implementation am lokalen Forward-Resolver oder am Public-Recursive-Resolver. Es ist damit selbst im Falle einer erfolgreichen Cache-Poisoning Attacke auf den Public Resolver nicht möglich RRs zu kompromittieren, die durch DNSSEC Signaturen geschützt sind.

\section{Aufbau}
\label{sec:architecture}
Zu Validierung des Entwurfs wurde ein einfacher Testaufbau umgesetzt. Dieser verwendet Windows 10\footnote{Microsoft Windows Version 1803 (Build 17134.285)} als Test-Client-Betriebssystem und Fedora 28\footnote{Linux 4.17.19-200.fc28.x86\_64} als Betriebssystem des lokalen Resolvers. Der Local-Resolver selbst wurden zwecks Vergleichbarkeit mit zwei verschiedenen Software-Paketen umgesetzt: Unbound\footnote{Version 1.7.3 (kompiliert mit OpenSSL 1.1.0h-fips)} und Knot-Resolver\footnote{Version 2.4.1}. Als Trusted-Public-Resolver wurde das Quad9-Projekt gewählt da es DoT anbietet und einer zu befürwortende Privacy-Policy\cite{Quad9Privacy} folgt. Darüber hinaus werden Domänen, die in Zusammenhang mit Schadsoftware stehen, vom Quad9-Resolver automatisch geblockt. Dieses Feature kann zwar als Zensur verstanden werden, durch die aktuelle Bedrohung durch DNS unterstützte Malware \cite{Alcoy2017} und die strikt auf ``Phishing, Malware, and Exploit-Kits'' konzentrierte Sperrliste\cite{Quad9FAQ} kann die Bedrohung durch Zensur der durch Malware vorübergehend untergeordnet werden.   
\begin{wrapfigure}{r}{0.5\textwidth}
    \begin{center}
    \includegraphics[width=0.48\textwidth,trim={5mm 8mm 5mm 8mm},clip]{Impl_Scenarios}
    \end{center}
    \caption{Darstellung der getesteten Szenarien}
    \label{img:impl-scenarios}
\end{wrapfigure}

Wie in Abbildung \ref{img:impl-scenarios} dargestellt wurden für den Performance- und Funktions-Test vier Szenarien gewählt. Dabei erfüllt nur Szenario 4 alle beschriebenen Schutzmechanismen, wobei Szenario 3 nur die Anonymität gegenüber des Public Resolver verletzt. Geht man von einem vertrauenswürdigen Anbieter dieses Resolvers aus, kann auf die Anonymisierung über den NAT-Server verzichtet werden. Wie man sieht, wird in jedem Fall von einem sicheren lokalen Netzwerk ausgegangen, mögliche Lösungen bei unsicheren Netzwerken wir in Kapitel \ref{chap:future} diskutiert. 

\paragraph{Public-Resolver über UDP (1)}
Durch die direkte Verbindung zwischen Stub-Resolver und Public Resolver wird auf jegliche Vertraulichkeit der Übertragung verzichtet, da der in Windows integrierte Stub-Resolver keine Form der Verschlüsselung unterstützt. Abgesehen davon, ergibt sich durch die minimale Anzahl an Zwischenstellen eine optimale Latenzzeit, die sich positiv auf die gesamte Performance auswirkt.

\paragraph{Lokalen Forwarding-Resolver über UDP (2)}
In diesem Fall wird ein Forwarding-Resolver im lokalen Netzwerk installiert. Da dieser das klassische DNS-Netzwerkprotokoll zur Kommunikation mit dem Recursive-Resolver verwendet, ergibt sich noch kein Schutz der Vertraulichkeit. Mit einer restriktiven Konfiguration und verschiedenen, von der Implementation abhängigen, Sicherheitsfeatures können jedoch gewisse Arten von Spoofing-Attacken ausgeschlossen werden. Befindet sich eine Firewall an der Netzwerkkante so kann diese, durch den Einsatz eines lokalen Resolvers, sehr restriktiv Eingestellt werden. Dies verhindert direkte Angriffe auf Stub-Resolver von extern. Ein weiterer Vorteil besteht beim Einsatz von ``Knot-Resolver'' da dieser bestimmte DNS Rebinding Attacken (siehe \ref{sec:attack-dnsrebind}) abwehren kann, indem er interne IP-Adressen in Antworten verbietet\cite{KnotResolverDocRebinding}.

\paragraph{Lokaler Forwarding-Resolver mit DoT (3)}
Wird die Verbindung zum Recursive-Resolver nun durch ein Verschlüsselungsprotokoll wie DoT (siehe \ref{sec:tec-dot}) geschützt, ergibt sich ein klarer Vorteil: Es besteht eine sichere Verbindung zwischen dem Forwarding- und Recursive-Resolver, was Sniffing-, Spoofing-, sowie MITM-Attacken auf diesem Wege ausschließt. Durch den Einsatz von verbindungsorientierten Protokollen und Verschlüsselung werden jedoch der steigende Ressourcenverbrauch und höhere Latenzzeiten merkbar.

\paragraph{Lokaler Forwarding-Resolver mit DoT und NAT (4)}
Fügt man dem Aufbau nun einen NAT-Server (siehe \ref{sec:tec-nat}) hinzu, erhält man den Vorteil der Anonymität gegenüber des Recursive-Resolvers. Da dies der einzige Vorteil des Aufbaus darstellt, ist die dadurch einhergehende Erhöhung der Latenzzeit gegen den Schutzbedarf abzuwägen.

\section{Tests und Messungen}
\label{sec:measurements}
Die Kontrolle der unter \ref{sec:architecture} beschriebenen Varianten wurde nun mithilfe der Programme dig\footnote{Auf Ubuntu Subsystem for Windows10; DiG 9.10.3-P4-Ubuntu} sowie ``DNS Benchmark''\footnote{von Steve Gibson Version 1.3.6668.0} durchgeführt. Gemessen wird die gesamte benötigte Umlaufzeit (Round-Trip-Time; RTT) 100 ungecachter, zufälliger Einträgen. Die gewählten Konfigurationen der DNS-Server sind in \nameref{chap:appA} zu finden und wurden nach den offiziellen Dokumentationen und darin enthaltenen Sicherheitsempfehlungen erstellt. Als ``Hardware'' wurden 3 virtuelle Maschinen (1 vCPU, 1GB RAM) auf einem vom Test-Client unabhängigen Host genutzt. Um eine optimale Vergleichbarkeit der einzelnen Varianten zu erreichen wurden die Tests der Performance von einem Test-Client auf die 3 Server simultan durchgeführt. Dabei wurden in 2 unabhängigen Läufen alle unterschiedlichen Varianten einer Software getestet. In einem dritten Lauf wurde die direkte Performance von 25 offener DNS-Resolvern (siehe \ref{chap:appB}) getestet und die 10 besten zur Errechnung der direkten Vergleichswerte herangezogen. Für die Auswertungen wurden immer nur Werte von ungecachten Einträgen herangezogen, da nur diese eine Kommunikation mit dem Recursive-Resolver verlangen. Die Ergebnisse der Tests sind unter Kapitel \ref{chap:results} angeführt. 



\chapter{Ergebnis}

\begin{draft}
\begin{markdown}
* Keine gute Privicy trotz Encryption weil kein padding und timing attacken (Siby und Shulman). 
* Einführung trotzdem sinnvoll da erhöhter Aufwand für Abhörende und 
\end{markdown}
\end{draft}

\chapter{Ausblick}

% Wenn DNSSEC große Durchdringung hat => MitM Problem weg
% Wenn T-DNS anklang findet DoS, Spoofing und Poisoning von Servers wenig problematisch
% Wenn DoT in OS und DoH in alle Browser integriert wird kein Problem mehr auf "letzter Meile" 

\lipsum

% Literaturverzeichnis
\bibliography{literatur} 
\bibliographystyle{ieeetr}
\clearpage

% Abbildungsverzeichnis
\listoffigures
\clearpage

% Tabellenverzeichnis
\listoftables
\clearpage

% Abkürzungsverzeichnis
\phantomsection
\addcontentsline{toc}{chapter}{Abkürzungsverzeichnis}
\chapter*{Abkürzungsverzeichnis}
\begin{acronym}[XXXXX]
    \acro{ABC}[ABC]{Alphabet}
    \acro{WWW}[WWW]{world wide web}
    \acro{ROFL}[ROFL]{Rolling on floor laughing}
\end{acronym}

%Anhänge
\chapter*{Anhang A}
\label{chap:appA}

\begin{lstlisting}
rc-update add iptables 
/etc/init.d/iptables save
\end{lstlisting}

\lstinputlisting[language=Bash,caption={test}]{code/set-fw.sh}

\lstinputlisting[language=Bash]{code/unbound.conf}

\end{document}