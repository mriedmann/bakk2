% !TEX encoding = IsoLatin2
\documentclass[Bachelor, BIC, german]{twbook}

\usepackage[T1]{fontenc}
% Hier kann je nach Betriebssystem eine der folgenden Optionen notwendig sein, 
% um die Umlaute korrekt wiederzugeben: utf8, latin, applemac
\usepackage[utf8]{inputenc}

\usepackage{lipsum}
\setlipsumdefault{1-4}

% Einstellungen der Class TWBOOK
\title{DNS Client Security\\Bedrohungsszenarien und Lösungswege}
\author{Michael Riedmann}
\studentnumber{1510258054}
\supervisor{FH-Prof. Dipl.-Ing. Alexander Mense}
\place{Wien}

\kurzfassung{% Wird vom Template eingefügt, kein Chapter oder so einfügen!

Zusammenfassung
\lipsum

% Alles andere schon fertig? Wenn nein, geh weg!}
% TODO: Schlagwörter der Zusammenfassung ausfüllen
\schlagworte{Schlagwort1, Schlagwort2, Schlagwort3, Schlagwort4}

\outline{%% spellcheck-language "en"

% Wird vom Template eingefügt, kein Chapter oder so einfügen!

Abstract

\todo[inline]{Alles andere schon fertig? Wenn nein, geh weg!}}
% TODO: Schlagwörter des Abstracts ausfüllen
\keywords{Keyword1, Keyword2, Keyword3, Keyword4}

% --------------------------------------------------------------------------- %
\begin{document}
\maketitle

\chapter{Einleitung}
Das Domain-Name-System (DNS) bietet eine einfache Möglichkeit zur Namesauflösung und damit die Grundlage für menschenles- und merkbare Namen in modernen Computernetzwerken. Außerdem dient es als verteile Datenbank für simple Informationen über Netzwerke und Hosts. Speziell im Internet nimmt dieses Service damit eine zentrale Rolle im Verbindungsaufbau zwischen den vernetzen Systemen ein. Betrachtet man nun das Netzwerkprotokoll des DNS genauer, stellt man fest, dass es ohne jegliche Ansprüche an Informationssicherheit konzipiert wurde. Dieser Umstand ist dem Alter, beziehungsweise der Historie des Systems geschuldet, entspricht den heutigen IT-Sicherheitsstandards jedoch in keiner Weise. Aufgrund dieser Tatsache, wurden in den letzten Jahrzehnten verschiedenste Ansätze zur Lösung dieses Problems entwickelt. Trotz dieses Umstands hat es bis zum heutigen Tag keiner der entwickelten Standards geschafft eine weitreichende Durchdringung zu erlangen. Obwohl das Bedürfnis nach Sicherheit über die Zeit stark gestiegen ist, trägt DNS aktuell eher zur Verschärfung der Lage als zu dessen Befriedigung bei.

Diese Arbeit bietet in Abschnitt \ref{chap:dns} eine kurze Einführung in die Funktionsweise von DNS und gibt eine Einführung in sicherheitsrelevanten Aspekten des Systems (\ref{sec:dnssecurity}). Kern dieser Arbeit stellt eine detaillierte Darstellung der aktuellen Sicherheitsprobleme und mögliche Lösungen, mit Fokus auf Endgeräte und User, dar. Dazu werden, in Kapitel \ref{chap:threads}, die wichtigsten Bedrohungen dargelegt. Um diese besser zu verstehen, werden die häufigsten Angriffsmethoden (in Kapitel \ref{chap:attacks}) beschrieben. Anhand dieser konnten allgemeine Lösungsvorschläge gemacht werden, welchen in Kapitel \ref{chap:solutions} näher behandelt werden. Abschließend werden diese Konzepte mit aktuelle verfügbaren Technologien (\ref{chap:technologies}) verglichen. Die Gesamtheit der erarbeiteten Informationen fließt in einem, unter \ref{chap:implementation} Beschriebenen, Testaufbau zusammen. Dieser soll einen praktischen Ansatz zur bestmöglichen Erfüllung der Formulierten Ziele unter Zuhilfenahme bestehenden Technologien darstellen. Die Resultate in Hinblick auf Performance und Erfüllung der Sicherheitsziele wird abschließend in Kapitel \ref{chap:results} vorgestellt und diskutiert.


\include{work}

\chapter{Ergebnis}
\lipsum

%\section{Allgemeine Empfehlungen}

%\subsection{DNSSEC}

%\subsection{DNS-over-TLS}
%\subsection{Server- / Netzwerkaufbau}
%\subsection{Netwerksetup}
%\subsection{Vertrauenswürdige Resolver / Upstream-DNS Server od. IDS}

% *Nur mit entsprechender Validierung des Zeilservers (DoT, etc.) weil sonst anfällig auf MitM, BGP-Hijacking, usw.*

%\section{Konkrete Konzepte}

%\subsection{Enterprise: local-only DNS-Resolver mit DNSSEC und IDS, HTTP-Proxy for Clients/internal Servers}

%\subsection{Privat/EPU: DNS-over-TLS  mit Trusted Resolver (z.B. Stubby and Quad9)}

% Keine gute Privicy trotz Encryption weil kein padding und timing attacken (Siby und Shulman). 

% Literaturverzeichnis
\bibliography{literatur} 
\bibliographystyle{ieeetr}
\clearpage

% Abbildungsverzeichnis
\listoffigures
\clearpage

% Tabellenverzeichnis
\listoftables
\clearpage

% Abkürzungsverzeichnis
\phantomsection
\addcontentsline{toc}{chapter}{Abkürzungsverzeichnis}
\chapter*{Abkürzungsverzeichnis}
\begin{acronym}[XXXXX]
    \acro{ABC}[ABC]{Alphabet}
    \acro{WWW}[WWW]{world wide web}
    \acro{ROFL}[ROFL]{Rolling on floor laughing}
\end{acronym}
\end{document}