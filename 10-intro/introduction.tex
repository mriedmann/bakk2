\chapter{Einleitung}
Das Domain-Name-System (DNS) bietet eine einfache Möglichkeit zur Namesauflösung und damit die Grundlage für menschenles- und merkbare Namen in modernen Computernetzwerken. Außerdem dient es als verteile Datenbank für simple Informationen über Netzwerke und Hosts. Speziell im Internet nimmt dieses Service damit eine zentrale Rolle im Verbindungsaufbau zwischen den vernetzen Systemen ein. Betrachtet man nun das Netzwerkprotokoll des DNS genauer, stellt man fest, dass es ohne jegliche Ansprüche an Informationssicherheit konzipiert wurde. Dieser Umstand ist seinem Alter geschuldet, widerspricht jedoch damit den heutigen IT-Sicherheitsstandards. Aufgrund dieser Tatsache, wurden in den letzten Jahrzehnten verschiedenste Ansätze zur Besserung dieser Situation entwickelt. Trotzdem hat es bis zum heutigen Tag keiner der entwickelten Standards geschafft eine weitreichende Durchdringung zu erlangen. Obwohl das Bedürfnis nach Sicherheit über die Zeit stark gestiegen ist, trägt speziell DNS aktuell eher zur Verschärfung der Lage als zu dessen Befriedigung bei.

Diese Arbeit bietet in Abschnitt \ref{chap:dns} eine kurze Einführung in die Funktionsweise von DNS und gibt eine Einführung in sicherheitsrelevanten Aspekten des Systems (\ref{sec:dnssecurity}). Den Kern stellt eine detaillierte Darstellung der aktuellen Sicherheitsprobleme und mögliche Lösungen, mit Fokus auf Endgeräte und User, dar. Dazu werden, in Kapitel \ref{chap:threads}, die wichtigsten Bedrohungen dargelegt. Um diese besser zu verstehen, werden die häufigsten Angriffsmethoden, in Kapitel \ref{chap:attacks}, beschrieben. Anhand dieser konnten allgemeine Lösungsvorschläge gemacht werden, welchen in Kapitel \ref{chap:solutions} näher behandelt werden. Abschließend werden aktuell verfügbaren Technologien (Kapitel \ref{chap:technologies}) vorgestellt, die zur Umsetzung dieser Konzepte herangezogen werden können. Die Gesamtheit der erarbeiteten Informationen fließt in einem, unter \ref{chap:implementation} Beschriebenen, Testaufbau zusammen. Dieser soll einen praktischen Ansatz zur bestmöglichen Erfüllung der Formulierten Ziele unter Zuhilfenahme bestehenden Technologien darstellen. Die Resultate in Hinblick auf Performance und Erfüllung der Sicherheitsziele wird abschließend in Kapitel \ref{chap:results} vorgestellt und in einem abschließenden Fazit diskutiert.
