\chapter{Sicherheit in der IT}
\label{chap:itsecurity}
Da sich diese Arbeit größtenteils mit Sicherheit beschäftigt, ist es notwendig dieses Thema bewusst einzugrenzen und mögliche Mehrdeutigkeiten vorab zu klären.

\section{Allgemeine Definitionen}
Um ein einheitliches Verständnis zu erzielen, werden in diesem Kapitel die wichtigsten Begriffe kurz definiert.

\paragraph{Informationssicherheit}
Wie das \ac{BSI} in seinem Glossar darlegt, hat Informationssicherheit den Schutz von Informationen als Ziel \cite{BSIGlossar}. Dieser Begriff ist bewusst nicht auf digitale Daten oder Computer beschränkt, sondern umfasst alle Arten von Informationen. Um Informationssicherheit zu erreichen werden drei Schutzziele definiert: Vertraulichkeit, Integrität und Verfügbarkeit.

\paragraph{Vertraulichkeit}
Informationsvertraulichkeit von Daten und Systemen ist gegeben, wenn keine unautorisierte Informationsgewinnung möglich ist \cite[p. 10]{Eckert2013}. Diese Definition spart bewusst die Möglichkeit auf Veränderung oder den Besitz aus. Es ist also durchaus möglich, blinde Änderungen an Daten zuzulassen oder eine unautorisierte Partei in Besitz der Daten kommen zu lassen, solange es nicht möglich ist die darin enthaltenen Informationen zu extrahieren. Dieses Szenario ist in den meisten verschlüsselten Netzwerkübertragungen durchaus der Fall und verletzt den Begriff der Vertraulichkeit noch nicht.    

\paragraph{Integrität}
Datenintegrität ist gewährleistet, wenn eine unautorisierte und unbemerkte Veränderung von Daten unmöglich ist \cite[p. 9]{Eckert2013}. Der Schwerpunkt liegt, speziell im Bereich der Netzwerktechnik, auf dem Erkennen von Veränderung, statt deren Verhinderung. Wird eine Veränderung zuverlässig erkannt, ist Integrität bereits gegeben. Das Gewährleisten der Integrität garantiert dabei keine Vertraulichkeit und umgekehrt.

\paragraph{Verfügbarkeit}
Verfügbarkeit behandelt die Nutzbarkeit eines Systems durch berechtigte Subjekte. Es darf somit nicht zu einer unautorisierten Verweigerung des Zugriffs kommen \cite[p. 12]{Eckert2013}. Dies beinhaltet z.B. einen hardwarebedingten Ausfall eines Systems, schließt aber bewusst den aktiven Eingriff von Außen nicht aus.
 
\paragraph{Authentizität}
Ein wichtiges, abgeleitetes Schutzziel stellt die Authentizität dar. Diese ist gegeben, wenn \enquote{die Echtheit und Glaubwürdigkeit des Objekts bzw. Subjekts, [...] anhand einer eindeutigen Identität und charakteristischen Eigenschaften überprüfbar ist}\cite[p. 8]{Eckert2013}. Der Vorgang der Feststellung der Authentizität wird Authentifizierung genannt. Die behauptete Identität wird somit unter Zuhilfenahme einer charakteristischen Eigenschaft verifiziert. Ein einfaches Beispiel ist die Eingabe eines Passwortes, bei dem die behauptete Identität, also die Benutzerkennung, über die charakteristische Eigenschaft, das Wissen es Passwortes, überprüft wird.  

\paragraph{Datenschutz}
Das Konzept des Datenschutzes (engl. Privacy) wird in verschiedenen Quellen unterschiedlich definiert. Im Zuge dieser Arbeit genügt eine vereinfachte Betrachtungsweise. Datenschutz kann als Schutz vor dem Missbrauch personenbezogener Daten verstanden werden, wobei die Definition personenbezogener Daten ebenfalls variiert. Relevant ist, dass es sich bei Daten die einen Rückschluss auf die Identität einer Person zulassen um personenbezogene Daten handelt. Dazu gehören explizit Adressen (inklusive IP-Adressen), sowie jede Form von einfach verknüpfbarer ``Online-Kennung''\cite{Schwenke2018}.     

\section{Vertrauen}
\label{sec:trust}
Vertrauen (engl. Trust) hat verschiedenste Bedeutungen, wird hier aber nur im Kontext der asymmetrischen Kryptografie verwendet. Das Vertrauen liegt in der Fähigkeit eines Merkmals, die Identität einer Person auch wirklich zu beweisen und baut damit auf der Definitionen von Authentizität auf \cite{Perrin2010}. Erlaubt man die thematische Spezifizierung, so beschreibt Vertrauen die Glaubwürdigkeit der Beziehung zwischen einem Merkmal und dem kryptografischen Schlüssel der ihn besitzenden Person. Um diese Vertrautheit herzustellen, existieren verschiedene Möglichkeiten. Diese werden hier als ``Prozesse zur Herstellung einer Vertrauensstellung'' (engl. Trust-Establishment-Process) bezeichnet. Es gibt dabei drei, für diese Arbeit wichtige, Methoden.

\paragraph{Direkt}
Die einfachste Methode um eine Vertrauensstellung zwischen zwei Parteien herzustellen, ist die direkte Methode. Dabei wird jede Partei über bekannte Merkmale von der jeweils andere authentifiziert. Dies passiert zum Beispiel beim persönlichen Überreichen einer Abschrift der jeweiligen Schlüssels. Die Authentifizierung kann in diesem Beispiel über visuelle, auditive oder haptische Merkmale erfolgen.

\paragraph{Trust-On-First-Use}
Das \ac{TOFU} Konzept folgt der Annahme, dass ein Angreifer nur ein begrenztes Zeitfenster besitzt. Dadurch ist es unwahrscheinlich, dass genau die aller ersten Kommunikation zwischen zwei Parteien abgefangen und manipulieren kann. Unter dieser Prämisse ist es sicher die Vertrauensstellung während des ersten Austauschs von Nachrichten herzustellen. Diese Methode ist Nutzenden von \ac{SSH} und \acs{HTTPS} mit selbstsignierten Zertifikaten durchaus geläufig. Wie festgestellt werden konnte, ist dieses Verfahren von der Verantwortlichkeit der Parteien abhängig und damit als leicht Angreifbar zu werten \cite{Wendlandt2008}.

\paragraph{Chain-Of-Trust}
In der \ac{CoT} wird die Idee des indirekten, transitiven Vertrauens eingeführt. Die bekannteste Umsetzung ist als \ac{PKI} bekannt und liefert die Basis für den sicheren Datenaustausch im \ac{WWW}. Die Verknüpfung eines Schlüssels mit einem charakteristischen Merkmal eines Subjekts wird dabei von einer dritten Stelle, der \ac{CA} beglaubigt. Diese Bescheinigung wird dabei über ein kryptografisch signiertes Zertifikat festgehalten. Will eine andere Partei die Echtheit eines behaupteten Merkmals prüfen, kann sie über eine Kontrolle dieses Zertifikats sicher sein, dass eine dritte Partei den Zusammenhang überprüft hat. Eine offene Frage stellt nun die Vertrauensstellung zwischen prüfender Partei und \ac{CA} dar. Dies kann eben über die Namensgebende \ac{CoT} beantwortet werden. Entweder es wurde bereits eine direkte Vertrauensstellung zwischen der \ac{CA} und der prüfenden Partei hergestellt oder die \ac{CA} weist selbst ein Zertifikat einer anderen Stelle vor. Wird bereits vertraut, ist damit auch das zu prüfende Zertifikat vertrauenswürdig. Herrscht noch kein Vertrauen, wird die \ac{CA} des CA-Zertifikats untersucht; der Vorgang wiederholt sich; es entsteht eine Kette. Der Ursprungspunkt einer Kette wird allgemein als Wurzel (engl. Root) bezeichnet \cite[p. 423 ff.]{Eckert2013}.  
