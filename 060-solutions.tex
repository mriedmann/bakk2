\chapter{Lösungsvorschläge}
\label{chap:solutions}
Aufgrund der in den vorherigen Kapiteln beschriebenen Problemen und Angriffen können insgesamt sechs Lösungsvorschläge erstellt werden. Diese sind als Empfehlungen zu verstehenden und werden bereits von verschiedenen, existierenden Technologien erfüllt. Auf diese wird im Kapitel \ref{chap:technologies} genauer eingegangen.

\section{Authentizität der Records}
\label{sec:solution-recordauth}
Wie in Kapitel \ref{sec:thread-auth} beschrieben wurde, ist die Authentizität der \ac{RR} ein zentrales Problem in DNS. Eine einfache Lösung stellt der Einsatz kryptografischer Signaturen dar. Mit diesen können alle Einträge bei der Erstellung signiert und dadurch geschützt werden. Werden die Signaturen bei der Abfrage mit übertragen, können sie vom Client validiert werden. Dies macht die Sicherstellung der Integrität und Authentizität der empfangenen Einträge möglich. Werden die Zonendateien nur bei Änderungen signiert ist die Zone außerdem vor ungewollter Veränderung zu schützen. Das notwendige Vertrauen, in die für die Signaturen verwendeten Schlüssel, kann hierbei auf verschiedenste Wege erreicht werden (siehe Kapitel \ref{sec:trust}).

\section{Authentizität der autoritativen Server}
Neben der Authentizität der RR selbst, spielt zusätzlich die Authentizität der autoritativen DNS-Server eine wichtige Rolle. Einige der angeführten Angriffe, nutzen den Umstand, dass Resolver keine Möglichkeit besitzt die Identität der Gegenstellen zu prüfen. 

Eine Lösung besteht in der Signatur der Records (siehe \ref{sec:solution-recordauth}), ist jedoch einfacher möglich: Formal genügt es eine kurzlebige, verschlüsselte Sitzung zwischen Resolver und autoritativen Server zu etablieren. Dies kann über klassischen hybriden Verschlüsselungsverfahren erreicht werden. Ein Problem stellt dabei der Aufbau einer zuverlässigen Vertrauensstellung dar (siehe Abschnitt \ref{sec:tec-dnscurve}).

\section{Authentizität des Resolvers}
In Kapitel \ref{sec:thread-auth} wurde erwähnt, dass es im DNS-Netzwerkprotokoll keine Möglichkeit gibt die Authentizität der Gegenstelle, im Speziellen des Resolvers, zu prüfen. Abhilfe für dieses Problem schafft einerseits das zuvor besprochene kryptografische Signieren der Einträge. Dieser Schutz scheitert hingegen oft an der praktischen Umsetzung, da dieser durch einen MITM-Angriff (siehe Abschnitt \ref{sec:attack-mitm}) überwindbar ist \cite{Bau2010}. 

Eine mögliche Lösung dieses Problem stellt das Einführen eines Authentifizierungsverfahren dar. In den meisten Fällen wird dafür eine auf asymmetrische Kryptografie basierende Methode eingesetzt. Wir schon in Abschnitt \ref{sec:trust} erwähnt, gibt es verschiedene Möglichkeiten das notwendige Vertrauen zu einem gegebenen Schlüssel herzustellen. Nach der Authentifizierung kann die Authentizität der einzelnen Nachrichten über Signaturen geprüft werden.

\section{Vertraulichkeit der Verbindung}
Wie in Abschnitt \ref{sec:thread-priv} dargelegt ist DNS-Privacy zu einem wichtigen Thema geworden. Die nachhaltige Lösung zur Sicherstellung der Vertraulichkeit ist, wie damals bei HTTP, der Einsatz einer geeigneten Transportverschlüsselungstechnologie. Auf welcher Ebene diese Stattfindet ist dafür grundsätzlich nicht relevant. Der Einsatz von etablierten Technologien wie TLS liegt dennoch nahe, obwohl für DNS zudem alternative Lösungen entwickelt wurden. 

\section{Trennen von Adresse und Anfrage}
\label{sec:goals-sourceanon}
Um die sensiblen Verbindungsdaten zu schützen ist es wichtig, die Anonymität der anfragenden Person zu wahren. Konkreter sollen Adressen, die mit der entschlüsselten Anfrage in Verbindung steht, nicht auf die anfragende Person zurückführbar sein. Durch den Einsatz von rekursiven Resolvern wird diese Kombination aus Adresse und Anfrage zwar vor den autoritativen DNS-Servern verborgen, verschiebt das Problem dabei lediglich auf den Resolver. 

Die Lösung besteht in der Trennung von Adresse des Clients und dessen entschlüsselter Anfrage bzw. Antwort. Um dies zu ermöglichen ist es notwendig, das der erste, vom Client direkt angesprochene Server, nicht in der Lage ist die echte DNS-Anfrage oder Antwort zu entschlüsseln. Die Aufgabe dieses Servers besteht darin, die wahre Adresse des Clients zu verbergen und die Anfrage von einer anderen Adresse aus an einen Resolver weiterzuleiten. Der Resolver muss dann in der Lage sein, die Anfrage zu entschlüsseln, ist aber nicht mehr in Kenntnis über die echte Adresse des Fragestellers. Damit ist das problematische Tupel aus Adresse und Daten aufgelöst, was zur Verbesserung der Privacy beiträgt \cite{Schmitt2018}.

\section{Verbindungbehaftete Protokolle einsetzen}
\label{sec:goals-trafficamp}
Wie in Abschnitt \ref{sec:attack-dosamp} beschrieben, wird DNS selbst als Verstärker (Traffic Amplification; DoS Amplification Attack) für DoS-Attacken eingesetzt. Es ist daher zum Schutz anderer Internet-Services essenziell, das DNS und dessen Implementierungen vor dieser Art der Ausnützung zu schützen.

Die einfachste Möglichkeit einer Lösung besteht im Ablösen des Verbindungslosen UDP zugunsten eines Verbindungsorientierten Protokolls wie TCP. Der notwendige Verbindungsaufbau könnte im Falle eines klassischen IP-Spoofings nicht abgeschlossen werden, was das Stellen einer Anfrage verhindern würde. Damit wäre DNS nicht mehr als Verstärker nutzbar. Dieser Vorteil ist direkt mit dem Nachteil verbunden, dass der Verbindungsaufbau viel Zeit in Anspruch nimmt und zusätzliche Ressourcen auf dem Server bindet. Wie festgestellten werden konnte, ist die Auswirkung auf die Performance in den meisten Fällen jedoch stark überschätzt \cite{Zhu2015}.

Neben TCP können Protokolle wie DTLS, trotz UDP Transportprotokoll, eingesetzt werden. DTLS baut eine Sitzung auf, was die Möglichkeit zum Einsatz als Verstärker ebenfalls erschweren würde, da der Server erst nach erfolgreichem Handshake Anfragen akzeptiert \cite{rfc6347}. 
