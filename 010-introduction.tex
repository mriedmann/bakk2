\chapter{Einleitung}
\ac{DNS} bietet eine einfache Möglichkeit zur Namensauflösung und damit die Grundlage für menschenlesbare Namen in modernen Computernetzwerken. Es dient als verteilte Datenbank für simple Informationen über Netzwerke und Hosts. Dieses Service nimmt somit eine zentrale Rolle im Verbindungsaufbau zwischen vernetzten Systemen ein. Betrachtet man das DNS-Netzwerkprotokoll genauer, stellt man fest, dass es ohne jegliche Ansprüche an Informationssicherheit konzipiert wurde, was nicht mit heutigen IT-Sicherheitsstandards vereinbar ist. Dementsprechend wurden in den letzten Jahrzehnten verschiedene Ansätze zur Verbesserung dieser Situation entwickelt, wobei keine Entwicklung eine weitreichende Durchdringung erreichen konnte.

Diese Arbeit beschäftigt sich mit der Analyse der aktuellen Situation und präsentiert einen möglichen, konkreten Lösungsvorschlag. Obwohl zu Beginn eine kurze Einführung in die Themen Informations- und IT-Sicherheit (Kapitel~\ref{chap:itsecurity}), sowie DNS (Kapitel~\ref{chap:dns}) gegeben wird, ist ein grundlegendes Verständnis der Bereiche Systemtechnik, Netzwerktechnik, IT-Sicherheit und Kryptografie vorausgesetzt. Der Inhalt dieser Arbeit ist daher an erfahrene SystemadministratorInnen, ohne spezielle Erfahrung im Bereich DNS-Sicherheit gerichtet.

Den Kern der Arbeit stellt eine detaillierte Darstellung der aktuellen Sicherheitsprobleme und mögliche Lösungen, mit Fokus auf Endgeräte und User, dar. Dazu wurde eine Analyse der wichtigsten Bedrohungen und bekannter Angriffe erstellt (Kapitel \ref{chap:threads}f). Basierend auf diesen werden logisch gefolgerte, sowie in anderen Arbeiten vorgeschlagene, allgemeine Lösungsvorschläge (Kapitel \ref{chap:solutions}) abgegeben und mit bestehenden Technologien verknüpft (Kapitel~\ref{chap:technologies}).

Anhand dieser Analysen konnte eine praktische Umsetzung ausgearbeitete werden (Kapitel~\ref{chap:implementation}). Diese wurde auf Funktionalität und Performanz getestet. Die abschließenden Ergebnisse sind in Kapitel~\ref{chap:results} zu finden und zeigen die Funktionalität, sowie die Performance der vorgestellten Lösung im Vergleich zu anderen Methoden.    
