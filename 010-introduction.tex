\chapter{Einleitung}
Das \ac{DNS} bietet eine einfache Möglichkeit zur Namesauflösung und damit die Grundlage für menschenles- und merkbare Namen in modernen Computernetzwerken. Außerdem dient es als verteilte Datenbank für simple Informationen über Netzwerke und Hosts. Speziell im Internet nimmt dieses Service somit eine zentrale Rolle im Verbindungsaufbau zwischen den vernetzen Systemen ein. Betrachtet man das Netzwerkprotokoll des DNS genauer, stellt man fest, dass es ohne jegliche Ansprüche an Informationssicherheit konzipiert wurde. Dieser Umstand ist seinem Alter geschuldet, widerspricht damit aber dennoch den heutigen IT-Sicherheitsstandards. Zur Erfüllung dieser Ansprüche wurden in den letzten Jahrzehnten verschiedene Ansätze zur Verbesserung dieser Situation entwickelt. Trotzdem hat es bis zum heutigen Tag keiner der entwickelten Standards geschafft, eine weitreichende Durchdringung zu erlangen. Obwohl das Bedürfnis nach Sicherheit über die Zeit stark gestiegen ist, trägt speziell DNS in seinem momentanen Zustand mehr zur Verschärfung der Lage als zu dessen Entschärfung bei.

Diese Arbeit beschäftigt sich mit der Analyse der aktuellen Situation und präsentiert einen möglichen Lösungsvorschlag. In Kapitel \ref{chap:itsecurity} werden zur Einführung in das Thema Informations- und IT-Sicherheit die wichtigsten Begrifflichkeiten vorgestellt. Trotzdem wird in diesem und folgenden Kapiteln von einem grundlegenden Verständnis der Bereiche Systemtechnik, Netzwerktechnik, IT-Sicherheit und Kryptografie ausgegangen. Der Inhalt dieser Arbeit ist daher an erfahrene SystemadministratorInnen, ohne spezielle Erfahrung im Bereich DNS-Sicherheit gerichtet. Um das notwendige Vorwissen im Bereich DNS zu vermitteln, wird in Kapitel \ref{chap:dns} eine kurze Einführung in die Funktionsweise von DNS gegeben. Darüber hinaus werden die sicherheitsrelevanten Aspekte des Systems (siehe Abschnitt \ref{sec:dnssecurity}) näher erläutert. 

Den Kern der Arbeit stellt eine detaillierte Darstellung der aktuellen Sicherheitsprobleme und mögliche Lösungen, mit Fokus auf Endgeräte und User, dar. Dazu werden, in Kapitel \ref{chap:threads}, die wichtigsten Bedrohungen behandelt. Um diese besser zu verstehen, sind in Kapitel \ref{chap:attacks} die häufigsten Angriffsmethoden beschrieben. Basierend auf diesen werden logisch gefolgerte, sowie in anderen Arbeiten vorgeschlagene, allgemeine Lösungsvorschläge abgegeben (Kapitel \ref{chap:solutions}). 

Um diesen Empfehlungen einem praktischen Nutzen zuzuführen werden in Kapitel \ref{chap:technologies} aktuell verfügbaren Technologien vorgestellt, die zur Umsetzung dieser Konzepte herangezogen werden können. Die Gesamtheit der erarbeiteten Informationen fließt in einem, unter Kapitel \ref{chap:implementation} beschriebenen, Test-Aufbau zusammen. Dieser stellt eine Möglichkeit zur Erfüllung der formulierten Ziele unter Zuhilfenahme bestehenden Technologien dar. Die Resultate mit speziellem Fokus auf Performance werden in Kapitel \ref{chap:results} vorgestellt und in einem abschließenden Fazit diskutiert.
