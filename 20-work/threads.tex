\chapter{Schwachstellen}
\label{chap:threads}

\section{Fehlende Integrität und Authentizität}
\label{sec:Thread-Auth}

Wie in den Anfängen des World-Wide-Webs und dessen Transportprotokoll HTTP ist auch bei der Konzeptionierung der DNS Protokolls keine Rücksicht auf Sicherheitsaspekte wie Integrität, Authentizität und Vertraulichkeit genommen worden. Da DNS nach RFC1035\cite{rfc1035} auf jede Form von Authentifizierung verzichtet, ist auch ein Prüfen der Identität der jeweiligen Gegenstelle, sowie der eigentlichen Nutzdaten, nicht vorgesehen. Dies birgt eine hohe Anfälligkeit auf Angriffe die Anfragen und Antworten gezielt verändern. Des weiterem besteht keine Möglichkeit die Quellenauthentizität der RR zu prüfen. Damit können Angreifer gefälschte Einträge in die DNS Datenbank einbringen und so die Adresse bestimmter Servicenamen gezielt verändern. Diese Umleitung kann dann als Basis für umfangreiche Angriffe genutzt werden.

\section{Vertraulichkeit}
\label{sec:Thread-Priv}

Das Internet als globale, vernetzende Technologie, rückt immer weiter in den Mittelpunkt der alltäglichen Kommunikation. Damit verbunden ist der steigende Bedarf nach Schutz der übertragenen Daten. Die IETF behandelt dieses Thema in verschiedensten RFCs, wobei Privicy einen der Schwerpunkte darstellt. Mit der 2013 veröffentlichten RFC6973 \textit{Privacy Considerations for Internet Protocols}\cite{rfc6973} wurde der Grundstein für die aktive Förderung von vertraulicher Kommunikation im Internet gelegt . 
Für DNS wurde die darauf aufbauenden RFC7626 \textit{DNS Privacy Considerations}\cite{rfc7626} erstellt, welche die Relevanz von DNS Privicy klar darlegt. Im speziellen wird auf die notwendige Unterscheidung zwischen der freien Zugänglichkeit der DNS Daten an sich und der Öffentlichkeit der Anfragen aufmerksam gemacht. Da DNS in im ursprünglichen Konzept nicht unterscheidet, sind die Anfragedaten für alle Zwischenstellen lesbar. Darüber hinaus existiert kein Schutz der Vertraulichkeit der Übertragung, dies macht auch Unberechtigten das mitlesen der Kommunikation einfach möglich. 
Diese Umstände werden durch die in RFC7871\cite{RFC7871} spezifizierte Technik \textit{EDNS0 Client Subnet} zusätzlich verschärft. Mit dieser Technologie können sensibler Informationen, wie das Client Subnetz, auch durch Resolver hindurch weitergereicht werden. Wie in der \textit{Privacy Note} der RFC beschrieben sind solche Vorhaben jedoch mit größter Vorsicht und auf keinen Fall gegen den Willen der Nutzenden einzusetzen. Der 2017 eingebrachte Entwurf zum einführen einer ClientID\cite{Licht2017} wurde zur Unterstützung des Privicy-Gedanken nicht weiter verfolgt.

\section{Verfügbarkeit}
\label{sec:Thread-DosAmp}

Normale Internet-Dienstleistungen weisen heutzutage einen hohen Vernetzungsgrad untereinander auf. Da diese Verbindungen stark von DNS abhängig sind, hat der Ausfall eines einzigen zentralen Dienstes schwerwiegende Auswirkungen auf alle anhängenden Dienste. Dieser Effekt zeigte sich zuletzt 2016 als einer der größten DNS-Anbieter Dyn, inc. von einer DoS-Attacke für mehrere Stunden lahmgelegt wurde \cite{Krebs2016}. DNS ist zwar nicht anfälliger auf DoS als andere zentrale Internet-Services, das Schadensausmaß bei einer erfolgreichen Attacke ist jedoch unvergleichbar hoch. Aus diesem Grund, ist DNS seit Jahren das meist angegriffene Internet-Service weltweit \cite{Alcoy2017}. 
Abgesehen davon befindet sich DNS auch als Trägertechnologie von DoS-Attacken auf Platz 1 \cite{Alcoy2017}. Die leichte Ausnutzbarkeit und schlechte Zurückverfolgbarkeit macht DNS neben dem \textit{Network Time Protokol} (NTP) zu den beliebtesten Verstärkungsmethoden. Es besteht darüber hinaus das Problem, dass DNSSEC die Situation bei steigender Verbreitung weiter verschärft \cite{Anagnostopoulos2013}\cite{VanRijswijk-Deij2014}.

