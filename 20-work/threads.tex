\chapter{Bekannte Probleme}
\label{chap:threads}

Das DNS als Gesamtsystem, sowie das ursprünglich spezifizierte Netzwerkprotokoll weisen einige Schwächen auf. Diese werden, angelehnt an die Schutzziele der Informationssicherheit (siehe Kapitel \ref{chap:itsecurity}), zu folgenden drei Problembeschreibungen zusammengefasst.   

\section{Fehlende Integritäts- und Authentizitätsprüfung}
\label{sec:thread-auth}

Wie in den Anfängen des World-Wide-Webs und dessen Transportprotokoll HTTP ist auch bei der Konzeptionierung der DNS Protokolls keine Rücksicht auf Sicherheitsaspekte wie Integrität, Authentizität und Vertraulichkeit genommen worden. Da DNS nach RFC1035\cite{rfc1035} auf jede Form von Authentifizierung verzichtet, ist auch ein Prüfen der Identität der jeweiligen Gegenstelle, sowie der eigentlichen Nutzdaten, nicht vorgesehen. Dies birgt eine hohe Anfälligkeit auf Angriffe die Anfragen und Antworten gezielt verändern. Des weiterem besteht keine Möglichkeit die Quellenauthentizität der RR zu prüfen. Damit können Angreifer gefälschte Einträge in die DNS Datenbank einbringen und so die Adresse bestimmter Servicenamen gezielt verändern. Diese Umleitung kann dann als Basis für umfangreiche Angriffe genutzt werden.

\section{Fehlender Schutz der Vertraulichkeit}
\label{sec:thread-priv}

Die DNS Sicherheitsvorfälle vergangener Jahre hat DNS-Privacy, als lange vernachlässigtes Thema, wieder ins Rampenlicht gerückt\cite{Greenwald2013}\cite{turkybbc2017}\cite{turkywp2018}. Die \ac{IETF} behandelt dieses Thema in der Arbeitsgruppe DPRIVE, welche sich ausschließlich mit DNS Privacy beschäftigt. Die 2013 veröffentlichten RFC6973 \textit{Privacy Considerations for Internet Protocols}\cite{rfc6973} bildet dabei den Grundstein für die aktive Förderung von vertraulicher Kommunikation im Internet. Für DNS wurde die darauf aufbauenden RFC7626 \textit{DNS Privacy Considerations}\cite{rfc7626} erstellt, welche die Relevanz von DNS Privacy klar darlegt. Im speziellen wird auf die notwendige Unterscheidung zwischen der freien Zugänglichkeit der DNS Daten an sich und die Öffentlichkeit der Anfragen aufmerksam gemacht. Darüber hinaus existiert kein Schutz der Vertraulichkeit der Übertragung, dies macht auch Unberechtigten das mitlesen der Kommunikation einfach möglich.

\section{Angreifbarkeit der Verfügbarkeit}
\label{sec:thread-dosamp}

DNS ist seit Jahren das meist angegriffene Internet-Service weltweit und befindet sich darüber hinaus auch als Trägertechnologie von DoS-Attacken auf Platz 1 \cite{Alcoy2017}. Das fehlen eines Handshakes beim Verbindungsaufbau und die schlechte Zurückverfolgbarkeit machen DNS, neben dem \ac{NTP}, zur beliebtesten Verstärkungsmethoden für DoS-Angriffe. Da DNS-Server in den meisten Fällen von anderen DNS-Servern angegriffen werden, bedroht diese Anfälligkeit die Verfügbarkeit des Systems selbst, sowie anderer Systeme im Internet \cite{Kambourakis2008}.