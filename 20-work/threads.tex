\chapter{Bekannte Probleme}
\label{chap:threads}

\section{Fehlende Integritäts- und Authentizitätsprüfung}
\label{sec:Thread-Auth}

Wie in den Anfängen des World-Wide-Webs und dessen Transportprotokoll HTTP ist auch bei der Konzeptionierung der DNS Protokolls keine Rücksicht auf Sicherheitsaspekte wie Integrität, Authentizität und Vertraulichkeit genommen worden. Da DNS nach RFC1035\cite{rfc1035} auf jede Form von Authentifizierung verzichtet, ist auch ein Prüfen der Identität der jeweiligen Gegenstelle, sowie der eigentlichen Nutzdaten, nicht vorgesehen. Dies birgt eine hohe Anfälligkeit auf Angriffe die Anfragen und Antworten gezielt verändern. Des weiterem besteht keine Möglichkeit die Quellenauthentizität der RR zu prüfen. Damit können Angreifer gefälschte Einträge in die DNS Datenbank einbringen und so die Adresse bestimmter Servicenamen gezielt verändern. Diese Umleitung kann dann als Basis für umfangreiche Angriffe genutzt werden.

\section{Fehlender Schutz der Vertraulichkeit}
\label{sec:Thread-Priv}

Das Internet als globale, vernetzende Technologie, rückt immer weiter in den Mittelpunkt der alltäglichen Kommunikation. Damit verbunden ist der steigende Bedarf nach Schutz der übertragenen Daten. Die IETF behandelt dieses Thema in verschiedensten RFCs, wobei Privacy einen der Schwerpunkte darstellt. Mit der 2013 veröffentlichten RFC6973 \textit{Privacy Considerations for Internet Protocols}\cite{rfc6973} wurde der Grundstein für die aktive Förderung von vertraulicher Kommunikation im Internet gelegt . 
Für DNS wurde die darauf aufbauenden RFC7626 \textit{DNS Privacy Considerations}\cite{rfc7626} erstellt, welche die Relevanz von DNS Privacy klar darlegt. Im speziellen wird auf die notwendige Unterscheidung zwischen der freien Zugänglichkeit der DNS Daten an sich und der Öffentlichkeit der Anfragen aufmerksam gemacht. Da DNS in im ursprünglichen Konzept nicht unterscheidet, sind die Anfragedaten für alle Zwischenstellen lesbar. Darüber hinaus existiert kein Schutz der Vertraulichkeit der Übertragung, dies macht auch Unberechtigten das mitlesen der Kommunikation einfach möglich. 
Diese Umstände werden durch die in RFC7871\cite{RFC7871} spezifizierte Technik \textit{EDNS0 Client Subnet} zusätzlich verschärft. Mit dieser Technologie können sensibler Informationen, wie das Client Subnetz, auch durch Resolver hindurch weitergereicht werden. Wie in der \textit{Privacy Note} der RFC beschrieben sind solche Vorhaben jedoch mit größter Vorsicht und auf keinen Fall gegen den Willen der Nutzenden einzusetzen. Der 2017 eingebrachte Entwurf zum einführen einer ClientID\cite{Licht2017} wurde zur Unterstützung des Privacy-Gedanken nicht weiter verfolgt.

\section{Angreifbarkeit der Verfügbarkeit}
\label{sec:Thread-DosAmp}

DNS ist seit Jahren das meist angegriffene Internet-Service weltweit und befindet sich darüber hinaus auch als Trägertechnologie von DoS-Attacken auf Platz 1 \cite{Alcoy2017}. Das fehlen eines Handshakes beim Verbindungsaufbau und die schlechte Zurückverfolgbarkeit machen DNS, neben dem \textit{Network Time Protocol} (NTP), zur beliebtesten Verstärkungsmethoden für DoS-Angriffe. Da DNS-Server in den meisten Fällen von anderen DNS-Servern angegriffen werden, bedroht diese Anfälligkeit die Verfügbarkeit des Systems selbst, sowie anderer Systeme im Internet \cite{Kambourakis2008}. Es besteht darüber hinaus das Problem, dass die mit DNSSEC eingeführte Erweiterung EDNS0, sowie DNSSEC selbst, das Verstärkungspotenzial weiter erhöht \cite{Anagnostopoulos2013}\cite{VanRijswijk-Deij2014}.
