\chapter{Lösungsvorschläge}
\label{chap:solutions}
Aufgrund der in den vorherigen Kapiteln beschriebenen Problemen und Angriffen können insgesamt sechs Lösungsvorschläge erstellt werden. Diese sind als Empfehlungen zu verstehenden und werden bereits von verschiedenen, existierenden Technologien erfüllt. Auf diese wird im Kapitel \ref{chap:technologies} genauer eingegangen.

\section{Authentizität der Records}
\label{sec:solution-recordauth}
Wie in Kapitel \ref{sec:thread-auth} beschrieben wurde, ist die Authentizität der RR ein zentrales Problem in DNS. Zur Lösung wurde schon früh der Einsatz kryptografischer Signaturen vorgeschlagen. Mithilfe dieser könnten alle Einträge bei der Erstellung signiert, zusammen mit den Signaturen gespeichert, übertragen und dann von der Gegenstelle validiert werden. Dadurch wird es möglich die Integrität und Authentizität der empfangenen Einträge eindeutig sicherzustellen. Werden die Zonendateien nur bei Änderung signiert und der private Schlüsselteil anschließend wieder entfernt, besteht sogar die Möglichkeit Zonen-Konfigurationen vor ungewollter Veränderung zu schützen. Das notwendige Vertrauen, in die für die Signaturen verwendeten Schüssel, kann dabei auf verschiedenste Wege erreicht werden (siehe Kapitel \ref{sec:trust}). Abgesehen davon kann die Integrität der signierten Nachrichten nur mit sicheren Signaturmechanismen, konkret die darin genutzten Hash-Algorithmen, sichergestellt werden\cite{Stevens2017}. 

\section{Authentizität der autoritativen Server}
Neben der Authentizität der RR selbst, spielt auch die Authentizität der autoritativen DNS-Server eine wichtige Rolle. Einige der angeführten Angriffe, nutzen den Umstand, dass Resolver keine Möglichkeit besitzt die Identität der Gegenstellen zu prüfen. Dies ist, aufgrund des zustandslosen Übertragungsprotokolls, selbst zwischen Anfrage und Antwort gegeben. Nach Standard verlässt sich ein Resolver ausschließlich auf die Kombination aus Adresse, Port und TXID um eine Antwort einer ausstehenden Anfrage zuzuordnen \cite{rfc1035}. Eine Lösung besteht in der Signatur der Records (siehe \ref{sec:solution-recordauth}), ist jedoch auch einfacher möglich: Formal genügt es eine kurzlebige, verschlüsselte Sitzung zwischen Resolver und autoritativen Server zu etablieren. Dies kann über klassischen hybriden Verschlüsselungsverfahren erreicht werden. Eine Problem stellt dabei der Aufbau einer zuverlässigen Vertrauensstellung dar (siehe Abschnitt \ref{sec:tec-dnscurve}).

\section{Authentizität des Resolvers}
In Kapitel \ref{sec:thread-auth} wurde erwähnt, dass es im DNS-Netzwerkprotokoll keine Möglichkeit gibt die Authentizität der Gegenstelle, im speziellen des Resolvers, zu prüfen. Abhilfe für dieses Problem schafft einerseits das zuvor besprochene kryptographische Signieren der Einträge, da so bei einem Eingriff keine unbemerkte Modifikation des Inhalts möglich ist. Dieser Schutz scheitert jedoch an der praktischen Umsetzung, da es einen Fallback-Mechanismus ausschließen müsste, da ansonsten eine Downgrade-Attacke möglich wäre\cite{Bau2010}. Bei diesem Angriff werden dafür die Signaturen aus dem Antwort-Paket entfernt, was den Eindruck erzeugt, der Server hätte die Unterstützung von Signaturen eingestellt. Die Nachrichten werden daraufhin ohne Prüfung der Signaturen abgearbeitet; Das System ist damit wieder angreifbar. 

Eine mögliche Lösung dieses Problem stellt das einführen eines Authentifizierungsverfahren dar. In den meisten Fällen wird dafür eine auf asymmetrische Kryptografie basierende Methode eingesetzt. Wir schon in Abschnitt \ref{sec:trust} erwähnt, gibt es verschiedene Möglichkeiten das notwendige Vertrauen zu einem gegebenen Schlüssel herzustellen. Nach der Authentifizierung kann die Authentizität der einzelnen Nachrichten über Signaturen geprüft werden.

\section{Vertraulichkeit der Verbindung}
Wie in Abschnitt \ref{sec:thread-priv} dargelegt ist DNS-Privacy zu einem wichtigen Thema geworden. Die nachhaltige Lösung zur Sicherstellung der Vertraulichkeit ist, wie damals auch bei HTTP, der Einsatz einer geeigneten Transportverschlüsselungstechnologie. Auf welcher Ebene diese Stattfindet ist dafür grundsätzlich nicht relevant. Der Einsatz von etablierten Technologien wie TLS liegt jedoch nahe, obwohl für DNS auch alternative Lösungen entwickelt wurden. 

\section{Trennen von Adresse und Anfrage}
\label{sec:goals-sourceanon}
Um die sensiblen Verbindungsdaten zu schützen ist es wichtig die Anonymität der anfragenden Person zu wahren. Konkreter sollen Adressen die mit der entschlüsselten Anfrage in Verbindung steht, nicht auf die anfragende Person zurückführbar sein. Durch den Einsatz von rekursiven Resolvern wird diese Kombination aus Adresse und Anfrage zwar vor den autoritativen DNS-Servern verborgen, verschiebt das Problem jedoch lediglich auf den Resolver. 

Die Lösung besteht nun in der simplen Trennung von Adresse des Clients und dessen entschlüsselter Anfrage bzw. Antwort. Um dies zu ermöglichen ist es notwendig, das der erste, vom Client direkt angesprochene Server, nicht in der Lage ist die echte DNS-Anfrage oder Antwort zu entschlüsseln. Die Aufgabe dieses Servers besteht darin die wahre Adresse des Clients zu verbergen und an die Anfrage von einen anderen Adresse aus an einen Resolver weiterzuleiten. Der Resolver muss dann in der Lage sein, die Anfrage zu Entschlüsseln, ist aber nicht mehr in Kenntnis über die echte Adresse des Fragestellers. Damit ist das problematische Tupel aus Adresse und Daten aufgelöst, was zur Verbesserung der Privacy beiträgt \cite{Schmitt2018}.

\section{Verbindungbehaftete Protokolle einsetzen}
\label{sec:goals-trafficamp}
Abgesehen von der Verfügbarkeit des DNS selbst, wird es, wie in Abschnitt \ref{sec:thread-dosamp} und \ref{sec:attack-dosamp} beschrieben, auch selbst als Verstärker (Traffic Amplification; DoS Amplification Attack) für DoS-Attacken eingesetzt. Es ist daher für zum Schutz anderer Internet-Services essenziell, dass DNS und dessen Implementierungen nicht lohnender für Traffic-Amplification Versuche sind als andere Services.

Die einfachste Möglichkeit einer Lösung besteht im Ablösen des Verbindungslosen UDP zugunsten eines Verbindungsorientierten Protokolls wie TCP. Der notwendige Verbindungsaufbau könnte im Falle eines klassischen IP-Spoofings nicht abgeschlossen werden, was das Stellen einer Anfrage verhindern würde. Damit wäre DNS nicht mehr als Verstärker nutzbar. Dieser Vorteil ist jedoch direkt mit dem Nachteil verbunden, da der Verbindungsaufbau viel Zeit in Anspruch nimmt und zusätzliche Ressourcen auf dem Server bindet. Wie festgestellten werden konnte, ist die Auswirkung auf die Performance in den meisten Fällen jedoch stark überschätzt\cite{Zhu2015}.

Neben TCP können auch Protokolle wie DTLS, trotz UDP Transportprotokoll, eingesetzt werden. DTLS baut eine Sitzung auf, was die Möglichkeit zum Einsatz als Verstärker ebenfalls erschweren würde, da der Server erst nach erfolgreichem Handshake Anfragen akzeptiert\cite{rfc6347}. 
