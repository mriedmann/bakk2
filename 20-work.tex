\chapter{Bedrohungen}

Die meisten DNS-Sicherheitsrichtlinien richten sind an Betreiber von extern gerichteten Installationen. Das Thema "DNS Client Security" wird wenn nur am Rande behandelt. Der nachfolgende Abschnitt befasst sich daher von allem mit Bedrohungen der Interaktion zwischen DNS-Resolver und rekursivem DNS-Server. 
Es gibt dabei zwei grundlegende Möglichkeiten einen DNS-Client anzugreifen: Über direktes kommunizieren mit der Netzwerkschnittstelle des Clients oder mittels Angriff auf einen Teil der Client-nahen Infrastruktur. Um die Auswirkung klar ersichtlich zu machen wird in \ref{sec:BedrohungenÜbersicht} eine tabellarische Zusammenfassung mit den verletzen Sicherheitskriterien (nach CIA-Triade) geben. 

\section{Direkte Angriffe}

Angriffe die das Verhalten einzelner DNS-Clients über eingreifen in deren Kommunikationsfluss beeinflussen, können als "direkte Angriffe" zusammenfasst werden. Diese Verfahren sind nur effizient wenn der Angreifer entweder passiv (Sniffing) oder aktiv (Man-in-the-middle) an der Netzwerkverbindung des Clients beteiligt ist. 

\subsection{DNS Sniffing}

Als Sniffing Angriffe werden, nach CAPEC, alle Arten von Angriffen bezeichnet die es ermöglichen Nachrichten zwischen mindestens 2 Parteien zu beobachten, mitlesen und/oder mithören. (https://capec.mitre.org/data/definitions/157.html). Im Kontext "DNS" hat dies eine spezielle Bedeutung da es, aufgrund des fehlenden Schutz der Vertraulichkeit, ermöglicht, alle Informationen der Anfragen und Antworten einzusehen. Somit ist das Kommunikationsverhalten der Opfergeräte leicht nachzuvollziehen, was zum Beispiel den Recon-Schritt der Cyber-Kill-Chain (siehe \ref{}) erheblich erleichtert. 

In IP-Netzen gibt es verschieden Möglichkeiten eine Sniffing Attacke durchzuführen: 

Eine Möglichkeit ist der Angriff der Netzwerkhardware. Bei einem Hub als Netzwerkverteiler erhalten immer alle angeschlossenen Teilnehmer alle Pakete. Es können daher die Pakete aller anderen an diesem Hub angeschlossenen Geräte einfach mitgelesen werden. Wird ein Switch einsetzt wird für jeden Anschluss (Port) die MAC-Adresse des angeschlossenen Geräts in eine Tabelle eingetragen. Es werden somit nur noch Pakete mit passender MAC-Adresse an die entsprechenden Ports weitergeleitet. Ein Angreifer kann bestimme Switches jedoch durch künstliche Füllen der MAC-Switching-Tabelle (MAC-Flooding Angriff) in ein, einem Hub ähnliches, Verhalten zwingen. Soll ein spezielles Endgerät mit bekannter MAC-Adresse angegriffen werden, kann bei anfälligen Switches auch ein MAC-Duplication Angriff durchgeführt werden. Dabei sendet der Angreifer Pakete mit der gefälschter MAC-Adresse des Opfers was den Switch dazu verleitet die Adresse auf zwei Ports einzutragen. Anfällige Switches senden daraufhin die Pakete des Opfers auch auf den Port des Angreifers.

Ist die Netzwerkhardware nicht angreifbar bleibt noch die Möglichkeit den Netzwerkstack der Client-Geräte zu attackieren. Hier kann eine Schwäche im "Address Resolution Protocol" (ARP) ausgenützt werden. Dieses Protokoll ist für die Auflösung von logischen Adressen (z.B. IP-Adressen) zu Adressen der Hardware (MAC-Adresse) verantwortlich (https://tools.ietf.org/html/rfc826). Da ARP stateless konzeptioniert wurde und auch keine Art von Authentifizierung verlangt kann eine als "ARP Spoofing" bekannte Attacke durchgeführt werden. Mit dieser kann die Zuordnung zwischen MAC-Adresse und logischer Adresse (z.B. IP-Adresse) im lokalen Netzwerk bewusst manipuliert werden. Damit ist es einem Angreifer leicht möglich sich als vertrauenswürdiger Host des Netzwerks auszugeben. Wird die MAC-Adresse des Netzwerk-Gateways mit der eines abhörenden Rechners getauscht ist auch ein umfangreiches Abhören des Netzwerkverkehrs möglich.

% Da es Ethernet-Broadcasts zur Kommunikation einsetzt, somit Verbindungslos ist und keinerlei Form von Authentifizierung verlangt, kann ein Angreifer jede MAC-Adresse als Antwort auf die Frage nach jeder beliebigen IP-Adresse senden. Dies setzt natürlich die selbe Layer-2 Broadcast-Domain und das fehlen entsprechender Schutzmaßnahmen voraus. Gelingt das Eintragen einer gefälschten MAC-Adresse, werden alle Pakete die an die IP-Adresse des Ziels gesendet werden an die Maschine mit der gewählte MAC-Adresse geschickt. Wird nun die Hardware-Adresse des default Gateway gegen die des Angreifers getauscht, ist es diesem Problemlos möglich den gesamten Netzwerkverkehr außerhalb des lokalen Subnetzes mitzuschneiden und zu verändern.  

Eine weitere Möglichkeit sich als gefälschter Gateway zu positionieren ist es das "Dymamic Host Configuration Protocol" (DHCP) mittels "DPHC-Spoofing" auszunutzen. Bei diesem Angriff wird ein eigener DHCP-Server im Subnet des Opfers positioniert. Dieser gibt zwar gültige IP-Adressen aus, verändert aber die default Gateway und/oder die DNS Server Einstellungen. Somit kann ein Host des Angreifers als Gateway oder DNS-Resolver zwischengeschalten werden und den Netzwerkverkehr kontrollieren. Da DHCP wie ARP auf Ethernet-Broadcasts ohne Authentifizierung setzt sind die Schwachstellen sehr ähnlich. Da der attackierte Client jedoch auch das Annahmen der gefälschten Sitzung über Broadcast verteilt ist das Erkennen etwas einfachen als vom ARP-Spoofing. 

Eine spezielle Methode zum umlenken der Netzwerkpakete stellt das ein Angriff auf das "Internet Control Message Protocol" (ICMP) dar. Dieses Protokoll dient als Grundlage für verschiedenste Unterstützungsaufgaben in IP-Netzen. Am bekanntesten ist die ICMP-Ping Funktion, mit der sich die Konnektivität von Hosts prüfen lässt. Obwohl ICMP oft auf diese Funktionalität reduziert wird beinhaltet es eine Vielzahl von teils wenig bekannten Aufgaben. Die, in Hinblick auf Sniffing, relevante Schwachstelle verbirgt sich dabei in der ICMP-Redirect Funktion, welche für Optimierung des Routing-Prozesses vorgesehen ist. 

Außerdem wird DNS hauptsächlich über UDP verwendet, was ein fälschen (spoofing) der Absenderadresse trivial macht. Da dadurch auch die Transaktions ID und das Absendeport bekannt wird, kann der Angreifer sofort ein gefälschtes Antwortpacket senden und somit den Eintrag der abgefragten Domäne im Resolver des Ziels beliebig verändern oder erweitern. Durch das setzen von hohen TTL Werten kann dieser Zustand lange über die physische Präsenz des Angreifers hinaus aufrecht erhalten werden.

% Sniffing und Spoofing gemischt?!

% Wenn sniffing des Netzwerks möglich, dann DNS kein problem, da keinerlei Privicy. Auch bei DNSSEC ein Problem. DNSCrypt, DNSCurve, DNS-over-TLS, DNS-over-HTTPS lösen diese Probleme. Noch kein nativer support in vielen OS (atm nur experimentell in Andoid)

\subsection{DNS Spoofing}

% * DNS Spoofing/Faking ermöglicht MitM
% * Speziell bei IOT da durch begrenzte Leistung oft keine Verschlüsselung. Bei HTTPS trotzdem Möglichkeiten durch Social-Engineering

\subsection{DNS Rebinding}

% Durch DoS von Resolvern oder DNS-Servern können kritischen Unternehmensdienste kurzfristig außer Betrieb genommen werden. Da die Verbindung zwischen diesen hoch vernetzten Diensten stark von DNS abhängt hat der Ausfall eines einzigen zentralen Dienstes (DynDns Vorfall) schwerwiegende Auswirkungen auf alle anhängenden Dienste.

\subsection{DNS Spoofing (Blindflug)} % BS?!
Da DNS, wie in \ref{sec:DNS} beschreiben, auf UDP aufsetzt, kann die Absenderadresse leicht gefälscht werden. Existiert nun eine Schwachstelle in der Implementierung des lokalen DNS-Resolver kann die TXID leicht erraten werden. 

\section{Indirekte Angriffe}

\subsection{Reliance Upon Transitive Trust}

% * Unbemerkte übernehme von Domänen möglich
% * Komplimitierung von Stakeholder-Diensten möglich
%   * Bei Websites  (HTTP od. uralt Browser mit Mixed Active Content) führt die komprimitierung eines einzigen Ressourcenservers zum Kompromittierung der gesamten Seite: XSS wird einfach möglich wenn z.B. eine JS-Datei eines Werbeanbieters in die Seite geladen werden kann.
%   * Wenn eine einzige aktive Ressource über http nachgeladen wird oder für TLS Attacken (Poodle, ) anfällig ist, kann die seite und somit der client angegriffen werden.
% * Bei nicht verschüsselten Netzwerkprotokollen (plain SMTP/POP3/IMAP, FTP, MQTT) kann die Verbindung vollständig übernommen werden.
% * In jedem Fall sorgt eine erfolgreiche Attacke zum Übergang der Verfügbarkeitskontrolle an den Angreifer (bis zum Erkennen das Problems und entfernen der eingeschläuschten Einträge)

\subsection{Name Collisions and Leaked Queries} 

% * Durch eigene interne TLDs (z.B. .local) kann es zu Kollisionen im globalen Namespace kommen (new TLDs).
% * Mögliche Kollisionen können bewusst ausgenutzt werden.
% * Durch falsch/schlecht konfigurierte DNS-Resolver können interne Anfragen zu externen DNS-Servern getragen werden -> Information Disclousure
% * Speziell bei "home-use" und ohne "LockDown" kann durch lokale Proxies und Resolver von "leakage" betroffen sein. Auch BYOD-Geräte speziell gefährdet wenn durch falsche Konfiguration DNS-Anfragen zum Auflösen internen Ressourcen an externe DNS-Server gestellt werden.

\subsection{C&C/Exfiltration über DNS Tunneling}

% * Mit allen "offenen" resolvern nutzbar
% * Bei "best-practice"-Einstellungen des resolvers sehr langsam
% * Nicht einfach zu erkennen
% * Für sehr kleine Datenmengen durchaus zuverlässig (C&C)
% * KillChain: Data Exfiltration, Controll

\section{Übersicht}
\label{BedrohungenÜbersicht}

\chapter{Probleme und Lösungsansätze}
% Noch nicht auf spezielle technische Lösungen eingehen!

\section{Authentizität der Records}
Wie in \ref{} beschrieben baut sich das DNS aus verschiedenen Zonen auf, die wiederum aus Einträgen (Resource Records; RR) bestehen. Obwohl das Format und der Aufbau der beschreibenden Zonen-Files klar spezifiziert ist, wurde in der urprünglichen DNS-Spezifikation keine Möglichkeit zur Prüfung der Authentizität festgeschrieben. Somit ist es weder für die Server-Software noch für die empfangenen Clients möglich die Herkunft der Einträge zu prüfen. Dieser Umstand stellt eine der zentralen Schwachstellen der aktuellen DNS Infrastruktur dar und ist Grundlage für verschiedenste Anfriffe wie DNS Spoofing und DNS Cache Poisoning.

Zur Lösung dieses Problems wurde schon früh eine Erweiterung des Standard um spezielle RRs vorgeschlagen. Mit diesen werden kryptografische Signaturen zu verschiedenen Einträgen, auf Anfrage, mitgeliefert. Dadurch wird es möglich die Integrität und Authentizität der empfangenen Einträge sicherzustellen. Werden die Zonendateien offline signiert, besteht sogar die Möglichkeit Server-Konfigurationen vor ungewollter Veränderung zu schützen. Dies kann durch das verwenden von asymmetrischer Kryptografie ermöglicht werden. Wobei sich, wie bei allen Signaturverfahren, noch die Frage des Vetrauens stellt. Die notwendige Vertrauensstellung kann dabei auf verschiedenste Wege hergestellt werden. Die gängigsten sind "Chain-of-trust", "Web-of-trust", "Shared Key" und "trust-on-first-use".

In manchen Spezielfällen kann die Integrität der beabsichtigten Anfrage leider trotz Signatur nicht garantiert werden. Einer dieser Angriffe basiert auf die Schwächen mancher Cypher-Suits mit unzureichend sicheren Hash-Algorithmen oder zu kurzen Schlüsseln (siehe Shattered). Eine andere Möglichkeit bietet das sogenannte "BitSquatting". Bei diesem Angriff werden zufällige Fehler im Speicher von Geräten ohne Fehlerresistenten Speichermodulen ausgenützt. Da es dadurch zu flaschen Anfragen kommt, gibt es auch keine Möglichkeit sich auf Protokoll- bzw. System-Ebene zu schützen. Die einzige effektive Lösung stellt der Einsatz von ECC-RAM Hardware dar.   

% Lösung für Authentizität der Records => Signieren der Records (offline/online; DNSSec, DNSCurve, DoT/DoH; Chain-Of-Trust)
% BitSquatting?! Signieren schützt nicht, da die Anfrage schon früh gegen das falsche System gestellt wurd.


\section{Authentizität der Endstelle}
Die in den Beginnen des Webs mit HTTP ist auf bei der Konzeptionierung der DNS Protokolls keine Rücksicht auf potenziell feindseelige Server rücksicht genommen worden. Da DNS bewusst auf jede Form von Authentifizierung verzichtet ist auch ein Prüfen der Identität der jeweiligen Gegenstelle nicht vorgesehen. Dies birgt jedoch eine hohe Anfälligkeit auf Man-in-the-Middle Attacken. Ein Angreifer dem es gelingt sich aus sicht des Netzwerks zwischen eine der Komponenten im Abfrageablauf zu positionieren hat komplette Kontrolle über den Informationsfluss zwischen den betroffenen Geräten. Sollte sich der Angreifer zwsichen Endgerät und Recursive DNS Server befinden, kann jede DNS-Interaktion nach belieben manimuliert werden. Solle es dem Angreifer gelingen sich vor einem Authoritativem DNS-Server zu stellen, können alle Antworten an die von diesem Server bereitgestellten Domänen manimuliert werden. Die verteilte Natur von DNS sorgt zusätzlich dafür, dass, Einträge für lange Zeit verändert werden können, ohne dass es eine einfache Form der remediation gibt.

Abhilfe für dieses Problem schafft einerseits die schon besprochene kryptographische Signierung der Einträge, da so selbst bei einem Eingriff keine unbemerkte Modifikation des Inhalts möglich ist. Dieser theoretischer Schutz scheitert jedoch an der praktischen Umsetzung, da es einen Fallback-Mechanismus ausschließen müsste, da sonst der Angreifer einfach die Signature aus dem Antwort-Paket entfernen könnte.

Die verbreitetste Lösung für dieses Problem ist wohl ein auf asymmertischer Kryptografie besierendes Authentifizierungsverfahren. Am einfachsten wird dafür ein nach X.509 Standard erstelltes Zertifikat mit passendem Schlüsselpaar verwendet. Die Authentizität der Endstelle wird Anhand eines Handshakes sichergestellt, indem der Server durch die Entschlüsselung eines mit dem im Zertifikat genannten Geheimtexts beweißt, dass er im Besitz des passenden privaten Schlüssels ist. Die Validität des Zertifikats und damit des öffentlichen Schüssels wir über eine Chain-of-trust sichergestellt. Dabei muss der Client einer Root Certificate Authority (CA) schon vor dem Verbindungsaufbau zum Server vertauen. Ist das Server-Zertifikat nun von eine Vertauenswürdigen CA signiert, wurde es von dieser geprüft und ist damit transitiv vertauenswürdig.

% Lösung für Authentizität der Endstelle => Verbindungsaufbau nur nach authentifizierung des Servers (über X509 Certs)

\section{ Vertraulichkeit der Übertragung}
% Lösung für Vertraulichkeit der Übertragung => Verschlüsseln der Übertragung (TLS, DNSCurve)

Die nachhaltige Lösung ist, wie damals auch bei HTTP, der Einsatz einer geeigneten Transportverschlüsselung. Auf welcher Netzwerkebene diese Stattfindet und wie die eigentliche validierung der Identitäten durchgeführt ist dafür grundsätzlich egal. Der Einsatz von etablierten Technologien wie IPSec oder TLS liegt jedoch nahe.


\section{Vertraulichkeit der Anfragen}
% Lösung für Vertraulichkeit der Anfragen => Mit DNS nicht möglich. Blockchain basierte ansätze möglich

\section{Nutzung als DoS-Amplifier}
% Lösung für Nutzung als DoS-Amplifier => Protokolle mit Handshake





\chapter{Technische Umsetzungen}
% Kurzbeschreibung, Löst welches Problem?, Vorteile/Nachteile, Verbreitung

\subsection{DNSSEC}

Wie in \ref{sec:DNSSecurity} ausführlich beschrieben, erfüllt des DNS System selbst keinerlei Schutzziele der Informationssicherheit. DNSSEC, als eine Sammlung an IETF-Spezifikationen, erweitert DNS mit dem Ziel die Integrität und Authentizität der Ressourceneinträge sicherzustellen (\cite{Arends2005}). Dies wird über die krypographische Signatur der Einträge erreicht. Um die Signaturen validieren zu können, wird eine Chain-Of-Trust genützt, die ihren Ankerpunkt im von der ICANN veröffentlichten DNSSEC-Root-Key besitzt. Der Umstand, dass die ICANN somit als "single point of trust" fungiert und somit den Grundgedanken der "Verteiltheit des Internets" unterwandert ist einer der Hauptkritikpunkte an DNSSEC. Des weiteren findet die Erweiterung aufgrund der langen Entwicklungszeit, der (verglichen zum DNS-Protokoll) hohen Komplexität und dem erhöhtem administrativem Aufwand, nur wenig anklang. Seit der finalen Veröffentlichen 2005 konnten zwar 90 Prozent der TLD Betreibenden zum signieren ihrer Zonen bewegt werden, die Verbreitung signierter 2LDs ist mit ca. 4 Prozent jedoch viel zu gering um Maßgeblich zur Sicherheit des globalen DNS beizutragen (http://rick.eng.br/dnssecstat/). Zusätzlich wurde das Schutzziel der Vertraulichkeit bewusst bei der Konzeptionierung ausgeklammert. Aufgrund verschiedenster Entwicklungen der nahen Vergangenheit ((Türkei, Deutschland, USA Prism, etc)) ist der Bedarf nach einer vertrauten Möglichkeit zur Auflösung von Namen im Internet stark angestiegen. Betrachtet man nun zusätzlich den Umstand, dass das DNSSEC Netzwerkprotokoll noch stärker als reines DNS, für DoS-Amplification genützt werden kann, ist klar warum diese Spezifikationssammlung so wenig positive Resonanz erhält.

% Problem mit Privacy ist, dass der ISP oder der Staat auch einfach SNI oder IP Kommunikation untersuchen kann. Damit lassen sich Verkehrsdaten ähnlicher Qualtität bilden. Privicy ist also nicht das beste Argument.
% In Conclusio Hinweis auf die Tatsache, dass die CoT zur Zeit die einzige, einfache (weil DNSCrypt auch) darstellt, um die Authentizität gut zu prüfen. 
% TODO: CIAA-Quarttet in DNSSecurity beschrieben https://www.eosgmbh.de/schutzziele-cia-und-ciaa

\subsection{DNSCurve}

Aufgrund der schlechten Akzeptanz und dem fehlenden Schutz der Vertraulichkeit veröffentlichte der US-amerikanische Kryptograph Daniel J. Bernstein (auch bekannt als DJB) eine eigene Lösung namens DNSCurve. Diese dient zur Sicherstellung einer vertraulichen und authentischen Kommunikation zwischen rekursiv auflösendem Server und den authentitiven Servers. Sie baut auf dem, von DJB entwickelten, elliptische Kurven-Kryptosystem Curve25519 auf und verwendet so, im vergleich zu DNSSEC, welches RSA einsetzt, elliptic curve cryptography (ECC), für asymmetrische Operationen. Außerdem wird der, ebenfalls selbst entwickelte, Poly1305 message authentication code (MAC) für die Verschlüsselung und Echtheitsprüfung der Nachrichten eingesetzt. Der Einsatz von ECC sorgt, im Vergleich zu RSA, für eine signifikante Verbesserung der Performance bei der Schlüsselaushandlung, da weit kürzerer Schüssel bei gleichbleibender Sicherheit eingesetzt werden können\cite{Gupta2002}. Der in RFC7905 spezifizierten Poly1305 Algorithmus ist ebenfalls auf Geschwindigkeit, unter Einhaltung der Sicherheitsansprüche, optimiert\cite{Bernstein2005}. Somit war DNSCurve, bei dessen Veröffentlichung 2009, DNSSEC, speziell im Bereich Performance, weit überlegen. Da der höhere Leistungsansprüche der Validierung bis Heute eine starkes Argument gegen den Einsatz von DNSSEC dargestellt, wurde DNSCurve, trotz des ungewöhlichen Algorithmen, durchaus positiv aufgenommen \cite{Henry2013}. Dies Überrascht, da DNSCurve, wie D. Kaminsky feststellt, ein konzeptionellen Problem im Bereich des "Trust Establishment" Prozesses ausweißt \cite{Kaminsky2011}. Das Protokoll verlässt sich beim Auffinden der öffentlichen Schüssel auf einen speziellen NS-Eintrag der zu Ziel-DNS-Domäne. Da DNSCurve ohne Erweiterungen des DNS-Protokolls auskommt, wird dieser Eintrag alleinig durch dessen spezielles Format aufgefunden. Darüber hinaus wird der Schlüssel in den NS-Eintrag selbst kodiert. Ein Angreifer in aktiven MitM Position, könnte somit, durch simples Umschreiben des Eintrags, eine Kommunikation über DNSCrypt unterbinden. Selbst wenn kein Fallback zur Klartextkommunikation erfolgt, besteht noch immer die Möglichkeit, ein eigenes Schlüsselpaar zu generieren und den öffentlichen Schüssel des Ziels gegen einen eigenen zu tauschen. In der Spezifikation scheint zwar die Möglichkeit zum Aufbau einer CoT auf, da eine zentrale Vertrauensinstanz bewusst fehlt, wird diese jedoch von Kaminsky als "ineffektiv" bezeichnet, da sie laut ihm keine zufriedenstellende Lösung des Problems "Key Management" birgt. Es ist somit fraglich, ob DNSCurve in der Lage ist die Vertraulichkeit und Authentizität von DNS sicherzustellen.

%Einfall: Das Problem lässt sich in mehrere Teil gleidern: Authentizität des Eintrags selbst, Authentizität der Antwort der Server, Authentizität der Antwort des Resolvers, Vertaulichkeit der Anfrage/Antwort zwischen Client und Recursive Server, Vertraulichkeit zwischen Recursive Server und Dns Servern, Vertraulichkeit gegenüber Operating der "Zwischenserver"   

\subsection{DNSCrypt}

Zusätzlich zu DNSCurve und DNSSEC wurde 2011 das Netzwerkprotokoll DNSCrypt entwickelt. Dieses ähnelt DNSCurve im grundlegenden Aufbau insofern es die von DJB entwickelten Algorithmen Curve25519 und Poly1305 zum Schlüsselaustausch und zur Überprüfung der Nachrichten verwendet und das DNS Netzwerkprotokoll ohne weiter Modifikationen für die Kommunikation einsetzt. Der größte Unterschied liegt darin, dass DNSCrypt die kommunikation zwischen Resolver und rekursivem DNS Server schützt. Um dies zu erreichen, werden kurzlebige Zertifikate eingesetzt, welche zur Verschlüsselung der Anfragen und Validierung der Antworten eingesetzt werden. Diese Zertifikate sind mit einem langlebigen Schüssel signiert, wobei der öffentliche Schüssel von jedem Client explizit in eine Liste an vertrauenswürdigen Schüsseln aufgenommen werden muss.\cite{Denis2016} Dies umgeht zwar die konzeptionelle Schwäche von DNSCrypt formal, ist jedoch ohne entsprechende Unterstützungsmethoden unpraktikabel. Eine in manchen Implementierungen verwendete Lösung nach dem "Trust-on-first-use"-Prinzip (TOFU) wird als nicht zuverlässig erachtet.\cite{Wendlandt2008} Somit stellt DNSCrypt, ähnlich wie DNSCurve, zwar eine sichere  Möglichkeit zum Übertragen von DNS-Anfragen und Antworten an den auflösenden DNS Server dar, vernachlässigt jedoch ebenfalls das zentrale Thema des "Schlüsselmanagements".    

\subsection{DNS-over-TLS}

Das Kernproblem der DNS Security besteht in der fehlender Authentizität und Vertraulichkeit der Übertragung. Diesen Umstand hat DNS mit vielen älteren Netzwerkprotokollen gemein, speziell im Internetumfeld ist HTTP, neben DNS, eines der verbreitetsten Protokolle. Als Lösung für HTTP wurde SSL bzw. TLS festgelegt, welches, nach seiner Spezifizierung 2000, das meist genützte Transportverschlüsselungsprotokoll der TCP/IP-Suite wurde. Es liegt also die Sicherheitsschwäche von DNS ebenfalls mit TLS als Transportprotokoll zu beheben. Das Problem besteht dabei in der ursprünglichen Intention DNS Verbindungslos auszulegen, ob das Protokoll so simpel und effizient wie möglich zu gestalten. Da TLS ein verbindungsorientiertes Protokoll (meistens TCP) als Trägerprotokoll verlangt, stehen diese Anforderungen im Widerspruch zueinander. Der Zeitaufwand für das Aufbauen einer Verbindung, zusammen mit dem erhöhten Ressourcenaufwand auf der Serverseite wurde lange Zeit als finales Gegenargument gegen den Einsatz von DNS über TCP (DNS-T) verwendet. Wie jedoch Zhu et at. 2015 feststellen konnte, ist dieses Argument für moderne Systeme nicht mehr zulässig \cite{Zhu2015}. Das anbieten von DNS über TLS (DNS-over-TLS; DoT) wurde in den IETF Dokumenten RFC7858\cite{Hu2016} und RFC8310\cite{Dickinson2018} spezifiziert und ist daher sein 2016 als sichere Methode zur Übertragung von DNS Anfragen und Antworten zwischen 2 Endpunkten zu betrachten. Die Aufgabe der Quellenauthentizitätskontrolle wurde auf das TLS Protokoll übertragen und folgt damit den selben Mechanismen die schon von HTTPS bekannt sind. Die über ein X509 Zertifikat authentisierende Gegenstelle wird im Zuge des TLS-Handshake über vorinstallierte Stammzertifikate von Zertifizierungsstellen authentifiziert. Die Vertrauensstellung zu den Zielservern ist daher implizit transitiv über das Vertrauen zur ausstellenden Stelle des Serverzertifikats hergestellt.        

\subsection{DNS-over-HTTPS}

Einen mit DoT vergleichbaren Ansatz wählt die im IETF Draft "DNS Queries over HTTPS" (DoH) beschriebene Technologie \cite{Mcmanus2018}. Diese Technik nützt HTTPS statt TLS als Trägerprotokoll und ermöglicht damit auch nativen Web Applikation die Auflösung von DNS Anfragen. Der Entwurf sieht zusätzlich die Möglichkeit einer "Server Push" Funktion vor, welche es DoH-DNS-Servers erlaubt, von sich aus Pakete an Clients zu senden. Es wird argumentiert, dass so der Auflösungsprozess beschleunigt werden kann, da zusätzlich zur eigentlichen Anfrage, zugehörige, andere Einträge mitgesendet werden können. Diese müssen dann nicht nochmals angefragt werden, sondern sind schon im Cache des Resolvers geladen. Zusätzlich soll es möglich sein, die bestehende HTTP-Caching-Infrastruktur für DoH-Anfragen mitzubenützen. Die Funktionsweise im Hinblick auf Vertraulichkeit und Authentizität unterscheidet sich nicht von DoT und ist somit grundlegend valide.

\section{QNAME minimisation}

\section{EncDNS & Oblivious DNS}

\section{Alternative Ansätze: Namecoin, GNS, RAINS}

\section{Übersicht}
% Eher in conclusio?
% Tabelle aus Paper "Grothoff, C., Wachs, M., Ermert, M., & Appelbaum, J. (2018). Toward secure name resolution on the internet. Computers and Security. https://doi.org/10.1016/j.cose.2018.01.018"

\section{Serverseitige Schutzmechanismen}

\subsection{DNS-Resolver}

% Wenn kein DNSSEC erzwungen wird (was noch schwer möglich ist) muss das risiko flascher Records akzeptiert werden. Über spezielle, vertrauenswündige Resolver kann das Risiko jedoch verringert werden.

\subsection{Fortify Internal Resolvers}

\subsection{Analyze Navigation Telemetry}

\section{Allgemeine Empfehlungen}

\subsection{DNSSEC}

\subsection{DNS-over-TLS}
\subsection{Server- / Netzwerkaufbau}
\subsection{Netwerksetup}
\subsection{Vertrauenswürdige Resolver / Upstream-DNS Server od. IDS}

% *Nur mit entsprechender Validierung des Zeilservers (DoT, etc.) weil sonst anfällig auf MitM, BGP-Hijacking, usw.*

\section{Konkrete Konzepte}

\subsection{Enterprise: local-only DNS-Resolver mit DNSSEC und IDS, HTTP-Proxy for Clients/internal Servers}

\subsection{Privat/EPU: DNS-over-TLS  mit Trusted Resolver (z.B. Stubby and Quad9)}