\chapter{Implementierung}
\label{chap:implementation}

\todo[inline]{schreiben! viel! Experiment! (mid/low priv)}

\begin{draft}
Aufbau eines lokalen Forward Resolvers auf Alpine Linux basis mit Unbound als Forward DNS Server.
DNS-over-TLS zu Trusted Resolvern mit DNSSEC Validation support und allen sinnvollen Privacy-Features enabled. 

Config und tests alles fertig, iptables config fehlt noch. Einfach: 22 TCP on LAN für management, 53 TCP/UDP für inbound DNS von LAN und 853 TCP richtung internet. 80/443 TCP outbound für updates. UDP 123 für NTP. ICMP Ping. Alles andere Drop inbound und outbound.

Aufteilung:
\begin{itemize}
    \item Was ist das Ziel?
    \item Wie ist der Aufbau?
    \item Warum wurden die Komponenten so gewählt?
\end{itemize}

Für die Ergebnisse: Wie ist die Performance (Messungen)?, Gegen was schützt der Aufbau?, Verbesserungspotenzial

\end{draft}
