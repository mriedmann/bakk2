\chapter{Fazit}
\label{chap:conclusion}
Fasst man nun die ersten Kapitel dieser Arbeit zusammen, kann gesagt werden, dass DNS als problematische Technologie von der IT-Security-Community erkannt wurde und seit vielen Jahren an verschiedensten Lösungsansetzen gearbeitet wird. Wie in Kapitel \ref{chap:threads} dargelegt wurde, sind die einzelnen Kernprobleme gut identifizierbar. Auch konkrete Bedrohungen durch Angriffe sind gut verstanden und über verschiedenste Methoden beleuchtet. Es konnte daher eine klare Aufzählung an Angriffskonzepten ausgearbeitet werden (siehe Kapitel \ref{chap:attacks}). Diese Darstellung der Gefahren für DNS-Clients wurde, zusammen mit bestehender Literatur, zur Erstellung eines Maßnahmenkatalogs (Kapitel \ref{chap:solutions}) verwendet. Dieser bot ein solides Grundgerüst zur Auswahl der in \ref{chap:technologies} näher beleuchteten Technologien. Abschließend konnte aufgrund dieser Auswahl ein praxistauglicher Aufbau getestet werden (Kapitel \ref{chap:implementation}). 

Die Ergebnisse der Performance- und Funktionstests bestätigten die Werte und Prognosen anderer Arbeiten und konnten zeigen, dass der Einsatz bestehender DNS-Sicherheits-Technologien durchaus möglich und tragbar ist. Durch das Aufkommen großer Trusted-Public-Resolver Projekte wie Google DNS, Cloudflare DNS und Quad9 wird eine performante und gleichzeitig sichere lokale DNS-Auflösung möglich und einfach zugänglich. 

Durch den Einsatz des in dieser Arbeit vorgestellten Konzepts lassen sich alle beschriebenen Sicherheitsprobleme von DNS vermeiden. Der Ressourcenaufwand wird dabei durch einen lokalen Resolver getragen. Da die Tests mit geringsten Hardware-Ressourcen (virtuell;1 vCPU; 1GB RAM) durchgeführt wurden, ist davon auszugehen, dass die Kosten für eine mögliche Umsetzung weit unter 100 Euro bleiben würden. Außerdem ist aufgrund des gewählten Modus von einem überaus geringen Wartungsaufwand auszugehen, das das Konzept zum sichern kleiner Netzwerke mit geringem Wartungsbudget optimal geeignet.

Der gemessene Performanceverlust bei nicht gecachten Anfragen beläuft sich auf maximal 50-100\% Erhöhung der minimalen Umlaufzeiten, wobei sich die durchschnittliche Erhöhung bei lediglich 5-9\% beläuft  (siehe \ref{chap:results}). Je nach Anspruch kann sich dies zwar kritisch auf die Gesamtperformance der Clients auswirken, da sich diese Werte ausschließlich auf nicht gecachte Anfragen beziehen, ist davon auszugehen, dass diese Verzögerungen bei herkömmlicher Nutzung nicht negativ auf die User-Experience auswirken.

\section{Diskussion}
Betrachtet man den Testaufbau, können verschiedene Kritikpunkte ausgemacht werden. 

\paragraph{Sicheres lokales Netzwerk}
Der offensichtlichste ist die in Kapitel \ref{chap:implementation} angemerkte Annahme, dass kleine Netzwerke selten von innen heraus angegriffen werden. Diese Hypothese konnte im Zuge dieser Arbeit nicht verifiziert werden und bleibt deshalb offen zur Diskussion. Geht man von einem unsicheren lokalen Netzwerk aus, ist die Kommunikation zwischen Client und lokalem DNS-Resolver weiterhin angreifbar. Dies ist auf die fehlende Unterstützung von sicheren Übertragungsprotokollen in aktuellen, in \acp{OS} integrierten Stub-Resolvern zurückzuführen. Der Einsatz von Local-Loopback Resolvern (wie in \ref{chap:implementation} beschrieben) wäre zwar eine mögliche Lösung, birgt jedoch verschiedene Probleme in der praktischen Realisierung.

\paragraph{Fingerprinting}
Ein anderer Punkt ist die Tatsache, dass die verwendeten Verfahren trotz starker Verschlüsselung anfällig auf Fingerprinting-Analysen zu sein scheinen\cite{Shulman2014}\cite{Siby2018}. Dabei werden zwei grundlegende Eigenschaften der Nachrichten untersucht: Größe und Timing. Es wird somit der Effekt ausgenutzt, dass der Zusammenhang zwischen frei beobachtbaren Eigenschaften signifikant genug ist um einen Rückschluss auf bestimmte Anfragen zuzulassen. Dies könnte ausreichen, um die Privatsphäre der betroffenen Personen zu verletzen. Der inzwischen in Knot-Resolver implementierte Standard ``EDN0 Padding''\cite{rfc7830} ist dabei in der Lage, die wahre Größe verschlüsselter Nachrichten zu verschleiern. Außerdem wirkt sich der Einsatz von Trusted-Public-Resolvern positiv auf die Privacy aus, da sie die Analysen erschweren\cite{Shulman2014}. Ob diese Technik die vorgestellten Angriffe verhindert, werden jedoch erst zukünftige Arbeiten zeigen. 

\paragraph{DNS und HTTPS}
Darüber hinaus kann argumentiert werden, dass die Gefahren eines Angriffs auf DNS überschätzt werden. Diese Argumentationen stützt sich oft auf die immer weiter fortschreitende Verbreitung von HTTPS. Wird eine Webseite unter HTTPS veröffentlicht, würde eine DNS Spoofing Attacke aufgrund des fehlenden Zertifikats scheitern. Betrachtet man nun aber zum Beispiel die in allen großen Browsern vertraute ``Let's Encrypt''-CA so stellt man fest, dass diese in ihrem Validierungsprozess eine sogenannte DNS-01-Challange anbietet. Durch diese ist es möglich, den Besitz einer Domäne über einen DNS-Eintrag zu beweisen. Es wäre einem Angreifer nun möglich aufgrund gezielter Angriffe auf den CA-DNS-Resolver ein valides Zertifikat zu erhalten. Danach stellt auch HTTPS keine Hürde bei einem großflächigen Angriff dar.

\section{Ausblick}
Die Entwicklung der letzten Zeit lässt vermuten, dass sich der Funktionsumfang von Stub-Resolvern in Betriebssystemen\cite{Wallen2018} und Browsern\cite{McManus2018a} in naher Zukunft stark ändern könnte. Allgemein kann festgehalten werden, dass die in dieser Arbeit beschriebenen Probleme durch eine direkte Implementierung von sicheren Übertragungsprotokollen im Stub-Resolver einfacher und performanter lösbar werden. Da dies jedoch nur von den \ac{OS}- und Browserherstellern angestoßen werden kann, ist dies keine aktuell durchführbare Lösung. Die Zeit wird zeigen ob das Thema DNS-Client-Security weitreichend genug ist, um die Aufmerksamkeit dieser Unternehmen und Gruppen zu gewinnen.  

