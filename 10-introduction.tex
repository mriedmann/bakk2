\chapter{Einleitung}
Das Domain-Name-System (DNS) bietet eine einfache Möglichkeit zur Namesauflösung und damit die Grundlage für menschenles- und merkbare Namen in modernen Computernetzwerken. Außerdem dient es als verteile Datenbank für simple Informationen über Netzwerke und Hosts. Speziell im Internet nimmt dieses Service damit eine zentrale Rolle im Verbindungsaufbau zwischen den vernetzen Systemen ein. Betrachtet man nun das Netzwerkprotokoll des DNS genauer, stellt man fest, dass es ohne jedliche Ansprüche an Informationssicherheit konzipiert wurde. Dieser Umstand ist dem Alter, beziehungsweise der Historie des Systems geschuldet, entspricht den heutigen IT-Sicheitsstandards jedoch in keinster weise. Aufgrund dieser Tatsache, wurden in den letzten Jahrzehnten verschiedenste Ansätze zur Lösung dieses Problems entwicket. Trotz dieses Umstands hat es bis zum heutigen Tag keiner der postulierten Standards geschafft eine weitreichende Durchdringung zu erlangen. Der gängige Weg DNS zu nutzen, hat sich, aus Sicht der Informationssicherheit, seit dessen Einführung nicht verändert.

% Warum DNS Client? 

Diese Arbeit bietet eine kurze Einfühung in die Funktionsweise von DNS (\ref{sec:DNS}) und gibt eine Übersicht der sicherheitsrelevanten Aspeketen des Systems (\ref{sec:DNSSecurity}). Kern dieser Arbeit stellt eine Analyse der gängigsten Angriffsmethoden auf DNS-Clients, sowie deren Mitigation, dar (\ref{sec:Attacks}). Abschließend werden zwei konkrete Aufbauten beschrieben, die unter den Gesichtpunkten Sicherheit, Wirtschaftlichkeit und Wartbarkeit als optimal anzusehen sind. 

\section{Namensauflösung}
% Zu oberflächlich? Besser nur als Satz in DNS ?!

\section{DNS}
\label{sec:DNS}

% ACHTUNG!!! KOPIE AUS "YANG ET AL"
All DNS data are stored in core data structure called a
Resource Record (RR), and each RR has an associated
name, class, and type. For example, an IPv4 address for
www.ucla.edu is stored in an RR with name www.ucla.edu,
class IN (Internet), and type A (IPv4 address). The set of all
RRs associated with the same name, class, and type is called
an Resource Record Set (RRset). Since DNS resolvers issue
queries for name, class, and type tuples, they are inherently
querying for RRsets (and not individual RRs). For example,
when a browser queries for hwww.ucla.edu, IN, Ai, the
reply will be the RRset for www.ucla.edu with all of the
IPv4 addresses for that name. Thus, the smallest unit that
can be requested in a query is an RRset, and all DNS actions
including cryptographic signatures discussed later, apply to
RRsets rather than individual RRs.
The global DNS is a distributed database organized in a
tree structure. At the top of the tree, the root zone delegates
authority to top level domains such as .com, .net, .org, and
.edu. The .com zone then delegates authority to create
ibm.com, .edu delegates authority to create ucla.edu,
and so forth. The information repository that makes up the
domain database is divided into sections called zones. Each
zone belongs to a single administrative authority and is
served by multiple authoritative name servers to provide
name resolution services for all names in the zone. By
definition, a zone can contain one or more connected
domains in the DNS name tree; in practice, many zones
contain only one domain—this is the case for top level
domains as well as large domains in general. In the rest of
this paper, we use the terms domain and zone interchangeably
when a zone contains a single domain.
Every RRset in the DNS belongs to a specific zone and is
stored at the nameservers of that zone. For example, the
RRset for hwww.ucla.edu, IN, Ai belongs to the
ucla.edu zone and stored in the ucla.edu nameservers.
Two important types of RRs, the NS RRs which hold the
names of DNS servers, and the corresponding A RRs which
hold the IP addresses of the NDS servers (called “glue
records”), play a critical role in establishing and maintaining
the DNS hierarchy. The NS RRset of each zone Z is
stored both locally and at the parent zone P, so that the
parent zone can refer the queries for Z’s DNS names to Z’s
DNS servers. When a zone changes any of its DNS servers,
it must notify its parent to update the NS RRset and A
RRset stored at the parent zone.
End users and applications resolve a DNS name by
querying the DNS for the corresponding RRset. Typically, a
simple stub resolver is implemented on every host which
sends DNS queries to a local caching resolver which takes the
responsibility of walking the DNS hierarchy to get the final
answer and then sends the answer back to the stub resolver.
% ACHTUNG!!! KOPIE AUS "YANG ET AL"

% DNS Server Types: Stub resolvers, Authoritative Servers, Recursive resolvers, forwarding dns servers
% Recursive Resolver position 
% Phase1 Phase2

% Wie funktioniert DNS?
% Grobe Übersicht max 2 Seiten

\section{DNS Security}
\label{sec:DNSSecurity}
\lipsum 

% Welche Probleme hat "klassisches" DNS? (Vertraulichkeit, Authentizität/Integrität)
% MitM (aLTEr, Crack, ARP-Poisoning, ...), Cache Poisoning, DNS Rebinding
% Transitiver Trust 

\section{Auswirkungen}
\lipsum

% was auch immer ich mir dabei gedacht hab X_x

\chapter{Verwandte Arbeiten}

\lipsum

%DONT! JUST DONT! Wenn was streichen das das!