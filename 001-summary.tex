% Wird vom Template eingefügt, kein Chapter oder so einfügen!

Das Ziel dieser Arbeit war es, Lösungsmöglichkeiten für aktuelle Sicherheitsprobleme der Domain Name Systems (DNS), vom Standpunkt des Client aus, näher zu untersuchen. DNS, als zentrales Namensauflösungs- und Informationssystem, stellt eine der wichtigsten Komponente des modernen Internets dar. Die Sicherheit des Internets ist dadurch eng an die des DNS gebunden. Mithilfe der bestehenden Literatur konnte eine umfangreiche Analyse der relevanten Bedrohungen und deren Mitigarionsmöglichkeiten durchgeführt werden. Basierend auf dieser wurde ein Lösungskonzept ausgearbeitet und über einen Test-Aufbau validiert. Die durchgeführten Funktions-Tests ergaben, dass der dargestellte Aufbau in der Lage ist, DNS Clients vor allen behandelten Bedrohungen zu schützen. Die Performance-Messungen ergaben eine Verschlechterung der durchschnittlichen Paketumlaufzeit von 5 bis 9\%. Diese Ergebnisse lassen darauf deuten, dass der vorgestellte Aufbau dazu in der Lage ist die Sicherheit maßgeblich zu verbessern, ohne eine starke Verschlechterung der User-Experience zu erzeugen.
